
\section{Data description\label{sec:Data-description}}

In this section, I describe my main data sources and the procedure
to build the dataset employed in my empirical analysis. Subsection
\ref{subsec:Data-Sources} lists the sources of the raw data. Subsection
\ref{subsec:KnowledgeMarkets} focuses on the definition of inventor
productivity measures and knowledge markets, which I identify through
realized inventor flows across sectors. Subsection \ref{subsec:Other data and aggregation}
briefly describes other data construction steps that are discussed
in more detail in Appendix \ref{app: Data-Construction-Details}. 

\subsection{Data Sources\label{subsec:Data-Sources}}

My empirical analysis relies on the variation of concentration across
product markets, as defined by 4-digit NAICS sectors, and the impact
of these shifts on the allocation of inventors with specific competences
across these sectors, and their effect on inventors' productivity.
I use USPTO patent data to measure inventor productivity and establish
the set of product markets that share similar inventors, and US Economic
Census data to obtain concentration and productivity growth measures.
Finally, I also use a dataset of product market regulations, Mercatus
RegData 4.0, to conduct an instrumental variable analysis, as well
as NBER-CES that I employ to obtain estimates of the Lerner Index
that I employ in the calibration of my theoretical model. 

My primary source is given by USPTO patent data from PatentsView.
This dataset contains disambiguated patent, inventor and assignee
identifiers, as well as Cooperative Patent Classification (CPC) classes
for each of the patents deposited in the period 1975-2021. I employ
these data to construct inventor flows across different sectors, which
I build employing the ALP classification of 1976-2016 patents into
NAICS sectors of application developed by \citet{goldschlagAlgorithmicLinksProbabilities2016}.
Since this classification is constructed using the PATSTAT dataset,
I rely on the crosswalk built by Gianluca Tarasconi to match these
two sources.\footnote{See \href{https://patentsview.org/forum/7/topic/143}{https://patentsview.org/forum/7/topic/143},
\href{https://rawpatentdata.blogspot.com/2019/07/patstat-patentsview-concordance-update.html}{https://rawpatentdata.blogspot.com/2019/07/patstat-patentsview-concordance-update.html}} This leaves me with around one third of all the patents registered
between 1975 and 2021, due to the restriction of the time frame to
1976-2016 as well as an incomplete match between PATSTAT and PatentsView.
Patent data also provides my with citation data, which I use to build
self-citations as well as truncation-corrected forward citations and
patent generality, following the procedure in \citet{hallNBERPatentCitation2001}
and \citet{acemogluRadicalIncrementalInnovationForthcoming}. I restrict
my attention to utility patents, since I am interested in patents
with a technological content, and not just design improvements.

My main data source for concentration and sales data is given by the
US Economic Census (EC), which reports sales shares for the top 4,
8, 20, and 50 firms, Herfindal-Hirschman Index, sales and number of
companies in various NAICS 4-digit sectors at a 5 year frequency.
I restrict my attention to the period between 1997 and 2012 for three
main reasons. First, as I show below, this period saw a substantial
increase in the concentration of inventors in specific technology
classes. Second, the start of this period coincides with an acceleration
in the growth of market concentration and markups \citep[see,  e.g., ][]{deloeckerRiseMarketPower2020}.
Third, 1997 marks the adoption of the NAICS classification, ensuring
a consistent definition of product markets throughout the period I
analyze, without the need to rely on crosswalk with the pre-existing
SIC classification. As my baseline concentration measure, I rely on
the HHI lower bound constructed by \citet{keilTroubleApproximatingIndustry2017}.\footnote{Available at \href{https://sites.google.com/site/drjankeil/data.}{https://sites.google.com/site/drjankeil/data.}}
This choice is dictated by the fact that the Economic Census reports
the HHI only for a subset of industries, which would severely limit
my sample. The method proposed by \citet{keilTroubleApproximatingIndustry2017}
obviates to this issue by constructing the implied lower bound of
the HHI implied by top sales shares reported in the Economic Census,
which are available for a much wider set of industries than the HHI.\footnote{As detailed in \citet{keilTroubleApproximatingIndustry2017} this
measure is very strongly correlated with the HHI reported by the Economic
Census when this is available, with a correlation of around 0.93.} While my estimates are robust to using the EC-reported HHI, this
choice allows me to obtain more power for my findings as well as generalize
them.

The Economic Census also provides me with sector-level growth in output
per worker, which constitutes my main measure of productivity growth
in the sectors I analyze. I choose this measure instead of multi-factor
productivity since the latter is available only for a limited set
of sectors, mostly in manufacturing. I deflate sales using NAICS-specific
price indices from the BLS.

I employ two additional data sources in the empirical analysis and
in the calibration of my model. First, I obtain sector-specific counts
of regulations for various NAICS 4-digit sector from the Mercatus
RegData 4.0 dataset, which I employ to conduct an instrumental variable
analysis, strengthening the causal interpretation of my results.\footnote{Available at https://www.quantgov.org/bulk-download.}
Second, I use NBER-CES data to produce estimates of the Lerner Index
following \citet{grullonAreUSIndustries2019a}.

Matching my various data sources results in a dataset of 157 NAICS
4-digit sectors over the period 1997-2012, out of a total of 304 business
sectors, for which I have data on concentration and that I merge to
my estimates of knowledge markets.

\subsection{Effective Inventors and Knowledge Markets\label{subsec:KnowledgeMarkets}}

In order to identify the effect of product market concentration on
the allocation of inventors, I first have to establish which set of
sectors different categories of inventors can actually choose from.
Indeed, pooling all sectors together would inevitably attenuate the
response of the inventor distribution to changes in the features of
individual product markets, as this design would correspond to the
assumption that all inventors could potentially be employed by any
sector. Since inventors have limited employment opportunities due
to their specific skills set, this assumption would incorrectly suggest
that, in a majority of cases, sector characteristics have no impact
on the allocation of researchers. That is, the mismeasurement of labor
market for inventors would produce an attenuation bias in my coefficient
of interest. 

The main aim of this section is correctly grouping product markets
that share the same \emph{required knowledge to innovate}, and therefore
compete for the same R\&D inputs, namely inventors. In practice, I
construct transition of inventors across different product markets,
and use the resulting network of flows to identify sectors that routinely
exchange researchers through a Louvain community-detection algorithm
\citep{blondelFastUnfoldingCommunities2008}. 

Figure \ref{fig: KmarketGraph} presents a graphical illustration
that clarifies my definition of knowledge markets as well as the procedure
I follow to construct them. Each node in the Figure represents a different
NAICS 4-digit sector, which is connected to other sectors by inventor
flows, indicated with black lines. The width of each line represents
the strength of these flows. The procedure described in detail in
the rest of this section, shows how I define these flows and measure
their strength. After obtaining these weighted flows, I employ a community
detection algorithm to group together sectors that are most closely
connected to each other. In the sectors that I selected for this illustration,
as one might expect, there are strong flows between grain, vegetable
farming and animal food manufacturing, all of which involve knowledge
related to agriculture and nutrition, and separately between footwear
and leather tanning, which both require knowledge of leather crafting.
In this case, my algorithm would identify two knowledge markets, one
given by the agriculture and food manufacturing sectors, and one given
by leather crafting sectors, leaving the laundry services sector isolated. 

This example makes clear the importance of identifying separate knowledge
market to analyze the effects of the evolution of product markets
on the allocation of inventors. In this figure, where there are routinely
almost no links of the laundry sectors to the others, we would not
observe a response of the inventor distribution to changes in the
concentration of these sectors, since the other inventors in the economy
specialize on different skills set and would therefore not be able
to transition there. Analogously, we would expect a higher profitability
in footwear manufacturing to attract inventors away from leather tanning,
but not from vegetable farming. 

\begin{figure}[t]
\caption{Graphical Illustration of Knowledge Markets}
\label{fig: KmarketGraph}
\begin{centering}
\includegraphics[width=0.9\textwidth]{\string"/Users/andreamanera/Dropbox (MIT)/Research/ProductInnovation/docs/JMP/draft/graphs/Kmarkets/Kmarkets.006\string".eps}\\
\par\end{centering}
{\footnotesize{}Note: This Figure provides a graphical illustration
of the definition of knowledge markets as sets of product markets
sharing the same required knowledge to innovate. This illustration
is based on transitions of inventors across product markets observed
in my data and classified in the same knowledge markets, although
many other sectors in these markets are excluded for the sake of exposition.
In the figure, nodes represent NAICS sectors 1111 (Oilseed and Grain
Farming), 1112 (Vegetable and Melon Farming), 3111 (Animal Food Manufacturing),
3161 (Leather and Hide Tanning and Finishing), 3162 (Footwear Manufacturing),
and 8123 (Drycleaning and Laundry Services). The edges connecting
nodes represent inventor transitions across sectors, while the width
of these edges represents the strength of the connection between the
two sectors as measured by undirected inventor flows.}{\footnotesize\par}
\end{figure}


\paragraph{Measuring Inventor Transitions}

The first step to construct knowledge markets consists in defining
transitions of inventors across sectors. I employ the USPTO patent
data classified into 4-digit NAICS sectors by \citet{goldschlagAlgorithmicLinksProbabilities2016}
to do so. Table \ref{tab:Data} exemplifies the structure of the USPTO
dataset matched with this NAICS classification. In this dataset, each
patent is listed together with all the inventors listed in the registered
patents. Each inventor is assigned a disambiguated ID corresponding
to the serial number of the first patent the inventor appears. In
this example, inventor 00001-1 and 00001-2 both cooperate on the development
of patent US00001. The third column in Table \ref{tab:Data} shows
an example of the \citet{goldschlagAlgorithmicLinksProbabilities2016}
classification for NAICS 4-digit industries. The authors conduct a
text analysis to classify each patent into the various NAICS sectors
of application of that patent. As the table illustrates, this classification
is not limited to a single sector per patent, and includes multiple
sectors in almost all instances. Importantly, this classification
captures the \emph{technological nature} of the patent and the sectors
of application of the knowledge required to develop that patent. While
other classifications, like the CPC or the USPC, also describe the
technological nature of patents, they do not allow a direct match
to sectors of application without arbitrarily assigning sectors to
each of these classes.

Given this data structure, I define a transition in two ways. First,
I consider inventor transitions \emph{within patents. }That is, I
consider that an inventor transition occurs between two sectors if
an inventor works on a patent that applies to both of these sectors.
The direction of flows does not matter for the definition of knowledge
markets, since I am interested in grouping sectors which exchange
researchers and not in the specific direction of these exchanges.
Therefore in the case of Table \ref{tab:Data}, I would say that there
have been two transitions between sectors 1111 and 1112 in 1980. Another
type of transition that I consider is \emph{across patents}. This
transition occurs when an inventor works on different patents that
apply to different product markets. An example of such transition
in Table \ref{tab:Data}, is the transition between sector 1112 and
3111 by inventor 00001-1. The raw count of transitions of inventors
across sectors in each year constitutes the basis of my measure of
inventor flows.
\begin{center}
\begin{table}

\begin{centering}
\caption{USPTO Data Structure}
\label{tab:Data}%
\begin{tabular}{|c|c|c|c|}
\hline 
Patent ID & Inventor ID & \citet{goldschlagAlgorithmicLinksProbabilities2016} NAICS & Year\tabularnewline
\hline 
US00001 & 00001-1 & 1111 & 1980\tabularnewline
US00001 & 00001-1 & 1112 & 1980\tabularnewline
US00001 & 00001-2 & 1111 & 1980\tabularnewline
US00001 & 00001-2 & 1112 & 1980\tabularnewline
US00002 & 00001-1 & 3111 & 1981\tabularnewline
\hline 
\end{tabular}\\
\par\end{centering}
{\footnotesize{}Note: This Table reports an example of the data structure
employed to build knowledge markets. The columns ``Patent ID'' and
``Inventor ID'' represent disambiguated patent and inventor identifiers
as classified by USPTO PatentsView Data. The column ``\citet{goldschlagAlgorithmicLinksProbabilities2016}
NAICS'' reports an example of patent classification into NAICS 4-digit
sectors. The data reported in this table have pure expository aim,
and do not represent actual observations in the dataset.}{\footnotesize\par}
\end{table}
\par\end{center}

\paragraph{Weighting Inventor Flows: Effective Inventors}

After identifying transitions, I proceed to weigh them by two alternative
measures in order to quantify the flow of inventors across sectors.
The first measure simply weigh each transition equally, computing
inventor flows as the raw count of researchers moving across NAICS.
The second measure adjusts for the productivity of individual inventors,
since raw counts might overstate or understate the importance of each
transition, depending on the size of origin and destination sectors,
their technological nature, as well as the ability of each inventor.
I therefore define a measure of ``effective inventors'' that aims
to correct for these and other omitted factors. For each inventor,
I estimate the fixed effect, $\alpha_{i},$ in the fully-saturated
regression,
\begin{align}
\text{\#Patents}_{cfit} & =\alpha_{i}+\gamma_{cft}+\varepsilon_{cfit},\label{eq:one-1-1}
\end{align}
where $\text{\#Patents}_{cfit}$ denotes the number of patents registered
in CPC class, $c,$ firm (assignee), $f$, and year $t$, that include
inventor $i.$ In this regression $\gamma_{cft}$ denotes a of CPC
class by firm (assignee) by year fixed effect. I choose to include
indicators for CPC classes at one digit, the broadest classification,
in order to identify as many fixed effects as possible. The inclusion
of CPC, firm, and year controls corrects for specific technological
features of the patented technology, the firm environment, as well
as the specific year. Further, this specification produces an estimate
of inventor productivity that accounts for the number of collaborators
on each patent. Clearly, these fixed effects might be inconsistently
estimated, and for this reason I check the robustness of all my results,
including the construction of knowledge markets, to the use of the
raw count of inventors rather than the inventor's productivity captured
by the fixed effect, $\alpha_{i}$. Given this specification, I define
an \emph{effective inventor} as one unit of the resulting fixed effects,
rescaled to take nonnegative values.

Armed with the results of this estimate, I define \emph{effective
inventor flows} between sector $j$ and sector $k$ at time $t$ as:
\begin{align*}
flow_{j\rightarrow k,t} & =\sum_{i}\#\left\{ \text{\ensuremath{i\text{'s}\ \text{transitions}\ j\rightarrow k\text{ \ensuremath{\text{in}} }t}}\right\} \cdot\alpha_{i},
\end{align*}
that is, the sum of transition counts weighted by effective inventors.
The total undirected flow between two sectors is then given by the
sum of inflows and outflows with ends in one of the two sectors:
\[
flow_{jk}=\sum_{t}\left(flow_{j\rightarrow k,t}+flow_{k\rightarrow j,t}\right).
\]
This flow measure defines a network of inventor transitions across
product markets, where the nodes, $j,k$, are given by 4-digit NAICS
codes, edges are given by transitions across sectors, and edge weights
are defined as a rescaled version of $flow_{jk}.$ I use these edge
weights as a measure of the strength of the connection between pairs
of sectors in the network. Rescaling the flow measure is necessary
in order to exclude effects of sector size as well as to avoid double
counting of inventors. I describe how I rescale this series in Appendix
\ref{app: Data-Construction-Details}.

\paragraph{Community Detection and Resulting Knowledge Markets}

I used rescaled the rescaled undirected flow measure as a network
edge weight to identify communities through the Louvain algorithm
developed by \citet{blondelFastUnfoldingCommunities2008}. This procedure
maximizes the modularity of the network choosing the number of communities
(knowledge market) and the assignment of nodes (NAICS sectors) to
communities. Modularity, a commonly used measure of connectedness
of networks, measures the distance between the density of links \emph{within
}communities versus \emph{between.}

This procedures produces 10 non-singleton sets of NAICS 4-digit sectors
that share the same inventors and have non-missing concentration measures.
By construction of the community detection algorithm, these knowledge
markets do not overlap, so each NAICS 4-digit sector belongs to one
and only one knowledge market. Figure \ref{fig: KmarketGraph-1} displays
the result of my procedure applied to NAICS 3-digit sectors. I report
this exercise since the 4-digit equivalent would be unreadable. However,
the knowledge markets identified by the two exercises are qualitatively
similar although they are clearly more numerous in the 4-digit case.
In this figure,{\footnotesize{} }lines denote inventor transitions,
with width proportional to the effective undirected inventor flow
between sectors. Nodes represent the NAICS 3-digit sectors reported
on each node. Black lines depict flows within knowledge markets, while
gray lines represent transitions between communities.

Three features are worth emphasizing. First, the network is very dense,
and transitions across 3-digit as well as 2-digit sectors are pervasive,
and differ largely in intensity. This points to the relevance of this
classification and its stark difference from what could be obtained
by grouping sectors based on broad product markets, which would neglect
the linkages across disparate markets, as well as pooling all sectors
together, which would neglect the difference in the strength of inventor
flows. Second, while the flows between communities might seem more
numerous than within communities, this is solely a by-product of the
circular layout of the network, whereby flows within close communities
on the circle are masked by the nodes. When applying the algorithm
to 4-digit sectors, I find that less than a third of flows occur between
communities, as expected since the community detection algorithm maximizes
the density of within-community linkages. Third, and perhaps most
importantly, the classification that I obtain is sensible, grouping
together sectors that we might expect to share similar knowledge to
innovate. Starting from sector 111 and going counter-clockwise, the
knowledge markets in the Figure can be described as follows. The first
market, including sector 111, groups sectors involving agricultural
production (111, 112 and 114) as well as food manufacturing (311).
The second market, starting with 211, includes oil, gas, and mining.
The green cluster at the top of the figure groups several heavy manufacturing
industries, like chemicals plastics and petroleum products, as well
as pipeline transportation (486). The market in yellow is concerned
mostly with transportation services and manufacturing as well as motor
vehicle dealers. The large blue cluster collects a large number of
retail sectors, as well as data processing, telecom and broadcasting
services. The remaining three markets include insurance and finance
(red cluster), computer, electronics, and machinery manufacturing
and professional services (violet), and wholesalers (gray).

Knowledge markets are identified using my measure of effective inventors,
but the algorithm produces nearly identical results when using raw
inventor counts; over 97\% of 4-digit NAICS pairs of sectors are classified
in the same manner using the two measures. That is, 97\% of sector
pairs belong to the same knowledge market according to both measures.

\begin{figure}[th]
\caption{Knowledge Markets Obtained from NAICS 3-digit Sectors}
\label{fig: KmarketGraph-1}
\begin{centering}
\includegraphics[width=0.8\textwidth]{\string"/Users/andreamanera/Dropbox (MIT)/Research/ProductInnovation/docs/JMP/draft/graphs/NewCommunity_3d\string".eps}\\
\par\end{centering}
{\footnotesize{}Note: This Figure displays the network of inventor
flows between NAICS 3-digit sectors and the knowledge markets resulting
from the application of the Louvain community detection algorithm.
Lines denote inventor transitions, with width proportional to the
effective undirected inventor flow between sectors. Nodes represent
the NAICS 3-digit sectors reported on each node. Black lines depict
flows within knowledge markets, while gray lines represent transitions
between communities.}{\footnotesize\par}
\end{figure}

\FloatBarrier

\subsection{Other Constructed Measures and Aggregation at Census Frequency\label{subsec:Other data and aggregation}}

\paragraph{Patent Citation Measures}

For each patent classified by \citet{goldschlagAlgorithmicLinksProbabilities2016},
I count the set of cited patents that belong to the citing patent's
assignee. In the case of cited patents with multiple assignees, I
consider half a count if the assignee is among them. The share of
self-citation is given by this count divided by total citations. I
construct five measures to correct self-citations for the assignee's
importance in the relevant technology class of cited patents. For
each citation made, excess self-citations are defined as $1-Pr\left(\text{self-citation}\right)$.
The various measures differ on how the probability of self-citation
is computed. For the first three measures, I compute this probability
as the assignee's share of total patents in the NAICS code attributed
to the citing patent. In employ in turn the share of NAICS patents
in the year, the previous five years, and the cumulative share from
the beginning of the sample. The other two measures are based on the
CPC classification at the group and subgroup levels (the lowest levels
of detail in the classification). For this measure, the probability
of self-citation is constructed for each citation by taking the share
of patents by the assignee in the CPC (sub)group and year corresponding
to the cited patents.\footnote{This procedure is similar to the approach followed in Akcigit and
Kerr's (2018) Appendix C.} \\
Finally, I aggregate all measures across assignees in the same NAICS
4-digit code using the number of patents in the relevant NAICS code
by each assignee in each year.

I also construct truncation-corrected two forward citation measures
and a patent generality measures following the definitions and the
procedures described in \citet{hallNBERPatentCitation2001}. The forward
citation measures compute the average number of citations received
by each firm's patents, giving an indication of the importance of
each patent for future technological developments. The correction
for truncation is conducted estimating the empirical CDF of the forward
citations lag distribution of patents in the relevant CPC 2-digit
technology class. The correction is then carried out by dividing the
overall amount of forward citations at the latest available date by
the inverse of the CDF thus obtained. The procedure suggested by \citet{hallNBERPatentCitation2001}
uses only information pertaining to the CPC 2-digit technology class
of the cited patent. In addition to this measure, I also conduct an
alternative correction that estimates a separate distribution for
each citing CPC 2-digit class and sums the corrected forward citations
across all citing classes. Patent generality also measures the technological
impact of patents, but rather than focusing on citations it examines
the scope of application of the patent. In particular, it computes
a measure of the dispersion of citation received across different
CPC classes. The higher is this dispersion, the wider is the technological
applicability of the patent.\footnote{The interested reader should consult \citet{acemogluRadicalIncrementalInnovationForthcoming}
for a detailed discussion, and the related appendix for details on
the construction of these measures. }

\paragraph{Regulation Data}

Mercatus RegData provides a count of restrictions imposed on a number
of NAICS 4d-digit product markets, obtained by matching a set of keywords
in NAICS descriptions to regulatory texts, and then taking the best
match for each document. However, the available data does not include
a set of codes for data quality reasons. \\
Therefore, I process the description of NAICS 4d codes and compute
the cosine-similarity between all pairs of sectors. I build an estimate
of sector-relevant restrictions for missing sectors by taking an average
weighted by cosine similarity of sectors included in RegData. In particular,
I include in the average the five most similar NAICS codes if similarity
is larger than .2, and I attribute the regulations of the most similar
sector otherwise. I chose this threshold by inspecting the similarity
associated to various NAICS pairs, and the assignment of regulations
to sectors is not highly sensitive to this choice.

\paragraph{Inventor Distribution Measures}

I employ the measure of effective inventors constructed as detailed
above to compute measures of researchers' concentration within sectors
for each year in my sample. Specifically, I use the PatentsView assignee
ID to identify firms that employ specific inventors in each sector,
and then compute several measures of the concentration of inventors
within sectors. I focus in particular on the top 10\% and bottom 50\%
share of inventors, as well as other commonly employed measures of
dispersion like the ratio of the $90^{th}$ quantile to the median.
I also compute the Gini coefficient of inventors across CPC classes
and NAICS 4-digit, assigning effective inventors to the relevant technology
class or NAICS sector, to document the trend in increasing concentration
of inventors in specific patent classes and sectors.

\paragraph{Aggregation at Census Frequency}

Data from the Economic Census are available at five-year frequency
for the years 1997-2017, which requires aggregating the other data
at the same frequency. Since I am interested in the effect of concentration
on the allocation of inventors, I average all variables related to
inventors and productivity using the five-year window \emph{starting
}in the census year (e.g., 1997-2001 for 1997), while I use concentration
measures for the related census year. In the IV regression I use product
restrictions as an instrument for concentration, which motivates me
to average restrictions in the five-year window \emph{ending} in the
census year (e.g., 1993-1997 for 1997). Since \citet{goldschlagAlgorithmicLinksProbabilities2016}'s
matching only covers the period up to 2016, I run all specifications
in long-differences over the time frame 1997-2012. 

\section{Empirical Analysis\label{sec:Results}}

This section presents four main findings that apply to the period
1997-2012: (i) effective inventors have become more concentrated across
economic sectors; (ii) sectors with increased concentration have attracted
a growing share of relevant inventor types; (iii) growth in the share
of relevant inventors is negatively correlated with inventor productivity,
as measured by forward citations as well as average growth in output
per worker divided by effective inventors employed; and (iv) growth
in the share of relevant inventors is positively correlated with the
share of self-citations and excess self-citations, as well as concentration
of inventors at the top within sectors. 

Results (i) and (ii) are indicative of a growing concentration of
inventors, and establish a positive causal link between the growth
in product market concentration and the increase in sectors' inventor
share. Findings (iii) and (iv) provide evidence in favor of misallocation.
Inventors' concentration in less competitive sectors turns out to
be inefficient, since there researchers are predominantly employed
on projects that do not contribute to the growth of the sector, which
I interpret as defensive innovation. This interpretation is supported
by the concentration of inventors among larger firms, the decline
in forward citations received by patents obtained by these firms,
and the decrease in growth per inventor that accompanies the increase
in product market concentration.

The rest of this section proceeds as follows. The first subsection
presents my empirical framework and variable definition. Remaining
sections present in order results (i)-(iv) above. I discuss the causal
interpretation of my results an IV specification in Subsection \ref{subsec:MainResults}.

\subsection{Variable Definitions and Main Specification}

The main outcomes of interest throughout the analysis refer to measures
of inventors concentration or measures of R\&D productivity. For the
former, I rely on the definition of effective inventors, $\alpha_{i}$,
provided in Section \ref{subsec:KnowledgeMarkets}, that is productivity-adjusted
inventors. For each product market, $p$, I define the share of inventors
employed by the sector in year $t$ as:
\[
\text{Inventor\ Share}{}_{p,t}\equiv\frac{\sum_{p(i,t)=p}\alpha_{i}}{\sum_{k(i,t)=k}\alpha_{i}},
\]
where the sum at the numerator is taken over all effective inventors
that are mentioned in patents registered in product market $p,$ while
the denominator computes the total effective inventors that belong
to the knowledge market. Effective inventors, $\alpha_{i},$ are also
the measure I use to evaluate the dispersion of inventors across sectors
and technology classes. My results are robust to computing the inventor
share using raw counts of researchers instead of effective inventors.

When analyzing R\&D productivity I focus on the three patent-based
measures described in Section \ref{subsec:Other data and aggregation},
that is, forward citations, share of self-citations, and patent generality.
Further, I compute a more direct measure of the productivity of inventors
given by the growth in output per worker divided by the number of
effective inventors employed by the sector.

In most specifications, the independent variables are measures of
concentration and controls for the size of the sector considered.
As discussed in Section \ref{subsec:Data-Sources}, my baseline measure
of concentration is the lower bound of the Herfindal-Hirschman Index
constructed by \citet{keilTroubleApproximatingIndustry2017} using
top sales share reported by the Economic Census for each sector.\footnote{The expression used to obtain this measure is: 
\begin{align*}
\text{\ensuremath{\underbar{HHI}}}_{p,t} & =4\left[\frac{\text{CR4}_{p,t}}{4}\right]^{2}+4\left[\frac{\text{CR8}_{p,t}-\text{CR4}_{p,t}}{4}\right]^{2}+12\left[\frac{\text{CR20}_{p,t}-\text{CR8}_{p,t}}{12}\right]^{2}+30\left[\frac{\text{CR50}_{p,t}-\text{CR20}_{p,t}}{30}\right]^{2},
\end{align*}
where ``CR\{X\}'' denotes the concentration ratio, that is the share
of sales, of the top X firms. This measure is a lower bound, and coincides
with the actual HHI if the sector has less than 50 firms, and sales
share are distributed equally in each of the top 0-4, 5-8, 9-20, 21-50
brackets. \citet{keilTroubleApproximatingIndustry2017} reports a
correlation of $\text{\ensuremath{\underbar{HHI}}}$ with the actual
index of 0.93.} I label this variable $\underline{\text{HHI}}_{p,t}$, where the
line below stands for the lower bound. This choice is motivated by
the small number of sectors in my sample that have an HHI index reported
by the Economic Census (about 80). Using the lower bound I can extend
my analysis to a total of 157 sectors. Due to the high correlation
between the two variables, the results are robust to using the Economic
Census HHI, as shown in robustness exercises. 

I obtain sales variable from the Economic Census, which I deflate
using BLS NAICS-specific price indexes. I use sales variables for
two purposes. First, real sales in 2012 as the weight in my regressions.
Second, I use the logarithm real sales as well as a quartic in real
sales in order to control for changes in the size of sectors during
my sample period. For the selected subset of sectors that reports
the number of establishments, I also explore the robustness of my
findings to controlling for sales per establishment. 

Given these definitions, my main specification is a sector-level long-difference
regression over the period 1997-2012:
\begin{equation}
\Delta\text{Share}_{p,\ 2012-1997}=f_{k}\mathbf{1}\left\{ p\in k\right\} \ +\ \beta\Delta\underline{\text{HHI}}_{p,\ 2012-1997}\ +\ \gamma'\Delta\text{Size}_{p,\ 2012-1997}\ +\ \varepsilon_{p},\label{eq: spec}
\end{equation}
where $\Delta\text{Share}$denotes the change in the inventors' share
of product market $p$, $f_{k}\mathbf{1}\left\{ p\in k\right\} $
is a dummy variable that takes value $1$ if the product market, $p$,
belongs to knowledge, $k$, $\Delta\underline{\text{HHI}}$ is the
change in the HHI lower bound, and $\Delta\text{Size}$ is a set of
controls for the size of sector $p$, which are given by either the
change in the logarithm of real sales or real sales per firm, or the
change in the terms of a quadratic polynomial in the real sales of
the sector. 

Regressions are weighted by sector sales in 2012 for the findings
which rely on Economic Census sector-level measures, and I estimate
robust standard errors in all specifications. When looking at patent
measures, I employ the same specification as \ref{eq: spec}, where
I replace the outcome variable with the change in patent productivity
and the dependent variable with the change in inventors' share. In
this case, since I do not rely on Economic Census measures, I report
unweighted regressions.\footnote{As I will show, the change in the inventor share is highly correlated
with the change in the HHI, so this specification essentially amounts
to a rescaling of the coefficient that would be obtained using the
HHI.} I also discuss the robustness of these findings to adopting the same
specification as \ref{eq: spec}, using the HHI lower bound and weighting
by sales. 

\subsection{Results\label{subsec:MainResults}}

\subsubsection{Inventor Concentration across NAICS Sectors has Increased}

Figure \ref{fig:Effective-inventor-HHI} reports the time series of
inventor concentration across NAICS 4-digit industries for the period
1976-2016, where the \citet{goldschlagAlgorithmicLinksProbabilities2016}
series is available. I construct these series using the share of effective
inventors (in panel (a)) or raw inventor counts (panel (b)). I use
the HHI index of inventor shares accruing to each sector as a measure
of concentration. Both panels display an increasing trend in the concentration
of inventors starting from the late 1990s. These patterns align closely
with the trends reported in \citet{akcigitWhatHappenedBusiness2019a},
which document a rising share of patents registered by top firms within
sectors. Figure \ref{fig:Effective-inventor-HHI} extends this findings
to the cross-industry allocation of inventors. Quantitatively, the
increase in inventor concentration is sizable, corresponding to about
a 20\% increase in the HHI for the effective inventor measure over
the period 1997-2012. Using raw inventor counts increases this number
to 30\%. 

As for the other results presented below, the effective inventor measure
and the raw inventor count behave similarly, albeit the series for
raw counts is more volatile and exhibits larger changes. This difference
in volatility stems from the regression method implied to obtain effective
inventors, which accounts for several features through time, firm,
and technology class fixed effects.

\begin{figure}[t]
\begin{centering}
\caption{Herfindal-Hirschman Index of Effective Inventors across NAICS 4-digit
Industries, 1976-2016 \label{fig:Effective-inventor-HHI}}
\par\end{centering}
\subfloat[Effective Inventors]{\begin{centering}
\par\end{centering}
\centering{}\begin{center}
\scalebox{.5}{\input{\string"/Users/andreamanera/Dropbox (MIT)/Research/ProductInnovation/docs/JMP/draft/graphs/FigureHHI.tex\string"}}
\par\end{center}}\subfloat[Inventor Count]{\begin{centering}
\par\end{centering}
\centering{}\begin{center}
\scalebox{.5}{\input{\string"/Users/andreamanera/Dropbox (MIT)/Research/ProductInnovation/docs/JMP/draft/graphs/FigureHHI_N_inv.tex\string"}}
\par\end{center}}\\

\raggedright{}{\small{}Note: This Figure reports the time series of
inventor concentration, as measured by the HHI index of inventor shares
across NAICS 4-digit sectors. The left panel reports the series constructed
using effective inventors as defined in Section \ref{subsec:KnowledgeMarkets},
the right panel uses instead raw inventor counts. Only the NAICS 4-digit
sectors with data for all years are included.}{\small\par}
\end{figure}
\FloatBarrier

\subsubsection{Markets with Growing Concentration Increased Their Inventor Share}

In this section, I present three sets of results for each specification,
which differ in the estimation sample to account for extreme observations.
In regression tables, ``Full Sample'' refers to the sample of observations
with non-missing observations for all the variables included. I propose
two sample selections to rule out that outliers drive the baseline
results. ``Trim Outliers'' refer to a sample which trims the most
extreme observations for the outcome and the independent variable
separately. I trim the observations that fall beyond three standard
deviations from the sample average of each variable, and that are
most likely to drive the results estimated using the full sample.\footnote{I justify the choices for each variable in detail in my replication
code using the empirical kernel density and detailed tabulations.} ``Mahalanobis 5\%'' denotes the sample where I trim the 5\% extreme
observations based on the Mahalanobis distance of pairs of observations
from the data centroid. Since this procedure is based on the joint
distribution of the outcome and independent variable, the sample thus
obtained varies in each regression.

Table \ref{tab: RegShInvNoctrl} presents the results of regression
(\ref{eq: spec}) where the outcome variable is the change in knowledge-market
inventor share, and the independent variable is the change in the
lower bound of the Herfindal-Hirschman Index discussed above, or the
index as reported by the Economic Census. The results in Table \ref{tab: RegShInvNoctrl}
highlight a strongly significant positive correlation between the
change in the HHI and the change in the share of effective inventors
accruing to each NAICS sector. Note that this regression is only partially
driven by the contemporaneous correlation between the two variables.
As discussed above, the share of effective inventors is average over
the five years \emph{starting }in the Economic Census year, while
the concentration measures refer to the Economic Census year only. 

Two important notes on the scale of the variables are in order. First,
here and in all following tables and graphs, all variables which refer
to shares or growth rates are reported in percentage points for ease
of interpretation. Therefore, for example the coefficient in Column
(1) of Table \ref{tab: RegShInvNoctrl} should be interpreted as saying
that an increase in one unit of the HHI index leads to an increase
in the share of the relevant knowledge market of $27.25$ pp. Second,
HHI indices are instead constructed to range between 0 and 1. In particular,
the HHI lower bound has sales-weighted an average of about .03, and
a standard deviation of .032 in 2012. According to Table \ref{tab: RegShInvNoctrl},
a standard deviation increase in this measure is associated to a 0.87
pp increase in the share of inventors accruing to the relevant NAICS
sector. In comparison, the sales-weighted average share of inventors
in 2012 is 1.18\%, with a standard deviation of 1.82\%, so the estimated
effect of a one standard deviation increase in concentration corresponds
to about half a standard deviation increase in the share of inventors
in the relevant market. Clearly, the estimates using the HHI lower
bound tend to be noisier as this is a constructed, and therefore imprecise,
measure of concentration. However, the number of available observations
is much larger than the actual HHI, allowing me to extend my findings
to a larger number of sectors (about double as can be seen from the
number of observations in each specification).\footnote{Regressions using the Economic Census HHI not reported in the main
text or the Appendix are available on request.}

While suggestive, the correlation presented above is far from causal,
as it neglects two fundamental components. First, it does not include
controls for the size of the sectors or firms, which could have a
confounding and mechanical effect on the share of scientists accruing
to a specific sector. Second, it estimates the correlation both across
and within knowledge markets. In Table \ref{tab: RegShInvHHI}, I
address these two limitations by restricting the analysis within knowledge
markets, and controlling for two measures of size. In the upper panel
of Table \ref{tab: RegShInvHHI}, I use the change in the logarithm
of real sales as a measure of the size of each sector, while in the
lower panel I present the results when average sales per firm are
included as a control. The inclusion of sales per firm is motivated
by the fact that there might be significant barriers to entry to R\&D,
easier to overcome for larger firms, mechanically linking concentration
and inventor hiring. Since the Economic Census reports the number
of companies only for a subset of firms, the sample used in the lower
panel is smaller than the upper panel. The results in Table \ref{tab: RegShInvHHI}
confirm the positive relation between the change in inventor shares
and concentration, and are largely unchanged relative to the estimates
in Table \ref{tab: RegShInvNoctrl}, suggesting that the correlation
does not arise mechanically from factors related to firm or sector
size. In particular, these findings imply that sectors with increasing
concentration have attracted a rising share of scientists above what
would be implied by their expansion in overall sales as well as average
firm size. 

Figure \ref{fig: scattersDksh} depicts graphically the residualized
observations underlying the estimated coefficients in Columns (2)
and (6) of Table \ref{tab: RegShInvHHI}, Panel (a). The upper panel
portrays changes of knowledge-market inventor shares over the change
in the HHI lower bound, after partialling out fixed effects for the
relevant knowledge market and changes in log real sales, with the
marker size proportional to the regression weight. Although the sample
displays some observations that appear extreme, the bulk of observations\textemdash and
especially weighted observations\textemdash falls on the regression
lines, mitigating the concerns that a few outliers might drive the
results. In any event, I explore the robustness of the results to
the exclusion of non-residualized observations, both manually and
defining extreme observations based on the Mahalanobis distance. Importantly,
this exercise reveals that the observations that appear extreme in
the residualized scatter are not unusual when considering the marginal
or joint distribution of non-residualized outcome and independent
variable. The bottom panel of Figure \ref{fig: scattersDksh} reports
the binned scatter plot corresponding to the sample where the 5\%
extreme observations according to the Mahalanobis distance have been
removed, and confirms that the positive relation between concentration
and inventor shares is not driven by a few extreme observations. In
particular, the corresponding regression results in Table \ref{tab: RegShInvHHI}(a),
Column (6), show that the estimated coefficient is significant at
a 5\% confidence level. The results presented in this section are
robust to using the raw number of inventors to compute the share of
researchers captured by each product market. 

Appendix Table \ref{tab: RegTotShare} shows estimates using the share
of effective inventors of each product market over the total. This
amounts to neglecting the fact that inventors flow only across sectors
that can employ their skills. In this specification, I find a significant,
albeit small, effect of product market concentration on the share
of inventors. However, this result only arises when the sample is
trimmed to remove outliers. This is unsurprising in light of the discussion
in Section \ref{subsec:KnowledgeMarkets}, whereby mismeasuring the
labor market for inventors should bias the estimates of inventor mobility
towards zero, since many of the sectors would actually not be routinely
connected by inventor flows. Finally, this result also conforms with
the findings in Table \ref{tab: RegShInvHHI}, which show that including
knowledge-market fixed effects does not alter the coefficients significantly,
suggesting that flows across knowledge markets are indeed negligible.

Appendix \label{app: Using-N-inv-1} establishes the robustness of
all the findings in this section to the use of raw inventor counts
rather than effective inventors to compute both inventor shares and
knowledge markets.

\begin{sidewaystable}
\caption{Regressions of Change in 4-digit Knowledge Market Share over Change
in HHI Measures, Long-Differences, 1997-2012\label{tab: RegShInvNoctrl}}

\begin{centering}
\scalebox{.9}{{
\def\sym#1{\ifmmode^{#1}\else\(^{#1}\)\fi}
\begin{tabular}{l*{6}{c}}
\hline\hline
                    &$\Delta$ Inventor Share (pp)   &               &               &               &               &               \\
                    &\multicolumn{1}{c}{(1)}   &\multicolumn{1}{c}{(2)}   &\multicolumn{1}{c}{(3)}   &\multicolumn{1}{c}{(4)}   &\multicolumn{1}{c}{(5)}   &\multicolumn{1}{c}{(6)}   \\
\hline
$\Delta \underline{\text{HHI}}$&      27.293*  &               &      27.183*  &               &      27.326*  &               \\
                    &    (11.569)   &               &    (11.941)   &               &    (11.620)   &               \\
$\Delta$ HHI        &               &      22.399***&               &      22.399***&               &      22.350***\\
                    &               &     (6.345)   &               &     (6.345)   &               &     (6.343)   \\
\hline
Knowledge Market FE &               &               &               &               &               &               \\
Sample              & Full Sample   & Full Sample   &Trim Outliers   &Trim Outliers   &Mahalanobis 5\%   &Mahalanobis 5\%   \\
Weight              &       Sales   &       Sales   &       Sales   &       Sales   &       Sales   &       Sales   \\
Observations        &         157   &          80   &         155   &          80   &         150   &          71   \\
\hline\hline
\end{tabular}
}
}\\
\par\end{centering}
\raggedright{}{\small{}Note: Regressions weighted by sales in 2012;
Robust standard errors in parentheses; Symbols denote significance
levels $\left(+\ p<0.1,^{*}\ p<0.05,^{**}\ p<.01,^{***}\ p<.001\right)$;
Checkmarks indicate the inclusion of fixed effects. This Tables presents
the results of specifications (\ref{eq: spec}), when the outcome
is the share of effective inventors of sector $p$ over total inventors
in knowledge market $k$, and the independent variable is the change
in the lower bound of the Herfindal-Hirschman Index for product market
$p$, as implied by Economic Census concentration ratios, or the HHI
index reported in the Economic Census. ``Full Sample'', ``Trim
Outliers'' and ``Mahalanobis 5\%'' refer to the samples described
in the main text.}{\small\par}
\end{sidewaystable}

\begin{sidewaystable}
\caption{Regressions of Change in 4-digit Knowledge Market Share over Change
in HHI Lower Bound, Long-Differences, 1997-2012\label{tab: RegShInvHHI}}

\begin{centering}
\subfloat[Controlling for Change in Log Real Sales]{\begin{centering}
\par\end{centering}
\centering{}\scalebox{.9}{{
\def\sym#1{\ifmmode^{#1}\else\(^{#1}\)\fi}
\makebox[\textwidth][c]{
\begin{tabular}{l*{6}{c}}
\hline\hline
                    &$\Delta$ Inventor Share (pp)  &               &               &               &               &               \\
                    &\multicolumn{1}{c}{(1)}   &\multicolumn{1}{c}{(2)}   &\multicolumn{1}{c}{(3)}   &\multicolumn{1}{c}{(4)}   &\multicolumn{1}{c}{(5)}   &\multicolumn{1}{c}{(6)}   \\
\hline
$\Delta \underline{HHI}$  &      26.093*  &      22.509*  &      25.904*  &      22.716*  &      26.111*  &      22.554*  \\
                    &    (10.696)   &    (10.848)   &    (11.124)   &    (10.948)   &    (10.725)   &    (11.019)   \\
$\Delta \log$ Sales  &       0.914** &       0.548*  &       0.881** &       0.539*  &       0.918** &       0.562*  \\
                    &     (0.278)   &     (0.243)   &     (0.275)   &     (0.242)   &     (0.283)   &     (0.261)   \\
\hline
Knowledge Market FE &               &   \ding{51}   &               &   \ding{51}   &               &   \ding{51}   \\
Sample              & Full Sample   & Full Sample   &Trim Outliers   &Trim Outliers   &Mahalanobis 5\%   &Mahalanobis 5\%   \\
Weight              &       Sales   &       Sales   &       Sales   &       Sales   &       Sales   &       Sales   \\
Observations        &         157   &         153   &         155   &         152   &         150   &         139   \\
\hline\hline
\end{tabular}
}
}
}}
\par\end{centering}

\begin{centering}
\subfloat[Controlling for Change in Log Real Sales per Company]{\centering{}\scalebox{.9}{ {
\def\sym#1{\ifmmode^{#1}\else\(^{#1}\)\fi}
\begin{tabular}{l*{6}{c}}
\hline\hline
                    &Ch. 4d K.M. Eff. Inv. Share (\%)   &               &               &               &               &               \\
                    &\multicolumn{1}{c}{(1)}   &\multicolumn{1}{c}{(2)}   &\multicolumn{1}{c}{(3)}   &\multicolumn{1}{c}{(4)}   &\multicolumn{1}{c}{(5)}   &\multicolumn{1}{c}{(6)}   \\
\hline
Ch. HHI lower bound &      35.230** &      20.783+  &      35.230** &      20.783+  &      35.154** &      22.854*  \\
                    &    (12.759)   &    (10.615)   &    (12.759)   &    (10.615)   &    (12.647)   &    (11.197)   \\
Ch. Log Real Sales per company&       0.175   &      -0.040   &       0.175   &      -0.040   &       0.300   &      -0.055   \\
                    &     (0.382)   &     (0.253)   &     (0.382)   &     (0.253)   &     (0.460)   &     (0.346)   \\
\hline
4D Knowledge Market FE&               &   \ding{51}   &               &   \ding{51}   &               &   \ding{51}   \\
Sample              & Full Sample   & Full Sample   &Trim Outliers   &Trim Outliers   &Mahalanobis 5\%   &Mahalanobis 5\%   \\
Weight              &       Sales   &       Sales   &       Sales   &       Sales   &       Sales   &       Sales   \\
Observations        &          81   &          79   &          81   &          79   &          75   &          67   \\
\hline\hline
\end{tabular}
}
}}\\
\par\end{centering}
\raggedright{}{\small{}Note: Regressions weighted by sales in 2012;
Robust standard errors in parentheses; Symbols denote significance
levels $\left(+\ p<0.1,^{*}\ p<0.05,^{**}\ p<.01,^{***}\ p<.001\right)$;
Checkmarks indicate the inclusion of fixed effects. This Tables presents
the results of specifications (\ref{eq: spec}), when the outcome
is the share of effective inventors of sector $p$ over total inventors
in knowledge market $k$, and the independent variable is the change
in the lower bound of the Herfindal-Hirschman Index for product market
$p$, as implied by Census concentration ratios. ``Full Sample'',
``Trim Outliers'' and ``Mahalanobis 5\%'' refer to the samples
described in the main text.}{\small\par}
\end{sidewaystable}

\begin{figure}
\caption{Residualized Scatter Plots Corresponding to Selected Columns in Table
\ref{tab: RegShInvHHI}, Panel (a)\label{fig: scattersDksh}}

\subfloat[Raw Scatter Plot, Specification in Column (2)]{\centering{}\includegraphics[width=0.9\textwidth]{\string"/Users/andreamanera/Dropbox (MIT)/Research/ProductInnovation/docs/JMP/draft/graphs/raw_Dk4_hhi_inv_fe10\string".eps}}

\subfloat[Binned Scatter Plot, Specification in Column (6)]{\centering{}\includegraphics[width=0.9\textwidth]{\string"/Users/andreamanera/Dropbox (MIT)/Research/ProductInnovation/docs/JMP/draft/graphs/bin_Dk4_hhi_inv_fe12\string".eps}}\\

\raggedright{}{\small{}Note: This figure presents residualized scatter
plots of the change in the share of effective inventors of sector
$p$ over total inventors in knowledge market $k$, over the change
in the lower bound of the Herfindal-Hirschman Index for product market
$p$, as implied by Census concentration ratios. The upper panel reports
the data for the full sample, where both variables are residualized
by change in log real sales and knowledge market fixed effects. The
size of the markers is proportional to the weight of each observation
in the regression (sector sales in 2012). The regression line uses
the coefficient on the change in HHI lower bound in Column (2) of
Table }\ref{tab: RegShInvHHI}{\small{}. The lower panel presents
a binned scatter plot removing the observations with the highest 5\%
Mahalanobis distance from the sample centroid. Observations are aggregated
using sales weights and the regression line is from Column (6) of
Table }\ref{tab: RegShInvHHI}{\small{}.}{\small\par}
\end{figure}


\paragraph{IV Results}

I now present IV results that suggest that the relation between concentration
and inventor shares is causal. Indeed, more concentration might just
be the result of an increase in technological entry barriers, established
by incumbents through an increase in their R\&D inventors. In this
scenario, the causality would flow from increased inventor shares
to higher concentration. Above, I tried to mitigate this concern using
the the average share of inventors following the Economic Census years
to which the HHI refers as my outcome variable. However, reverse causality
could still be present if the autocorrelation of inventor shares is
sufficiently high. This motivates me to produce 2SLS estimates that
instrument the change in the HHI lower bound with changes in product
market restrictions, as measured by the Mercatus dataset RegData 4.0.
Theoretically, an increase in restrictions should raise barriers to
entry in affected product markets, thus leading to higher concentration.
As discussed below, this proves to be the case empirically, making
a case for the validity of restrictions as an instrument for concentration.
A violation of the exclusion restriction requires a causal connection
between product market regulations and the share of inventors hired
by each sector, which acts independently of product market concentration.
A possibility in this sense is the increase in the number of inventors
required to fulfill product market restrictions, if such regulations
specifically affect technologies currently in use in the industry.
However, this effect should be both large and persistent to be captured
by my measure of inventor shares. Further, while RegData certainly
include product restrictions, there are also a number of regulatory
burdens that are not related to technological components, like reporting
obligations and other legal burdens. In addition, while product restrictions
might certainly induce a change in the direction of innovation, there
is no a priori reason to believe that the scale of innovation activity
should also increase. These considerations lead me to believe that
the exclusion restriction is not highly likely to be violated.

The results of the 2SLS estimation are presented in the upper panel
of Table \ref{tab: RegIV}. The specification is the same as in Column
(2) of \ref{tab: RegShInvHHI}, including both knowledge market and
sale change fixed effects. The 2SLS estimates confirm the significance
of concentration changes for the increase in knowledge market inventor
shares. The magnitudes of estimated coefficients are statistically
indistinguishable from the ones reported in the baseline regression.
The first-stage F clearly indicates that instruments are weak. This
is unsurprising since, as detailed above, both the HHI lower bound
and the regulation measures are constructed and therefore imprecise.
In particular, I had to impute regulations for a large part of the
sample using the cosine-similarity between product market restrictions.\footnote{Using only available sectors requires dropping two thirds of the observations.
See Appendix \ref{app: Data-Construction-Details} for details on
data construction. } However, instruments are not irrelevant. The results in the lower
panel of Table \ref{tab: RegIV} imply that the first-stage t-statistic
for the regression of the change in the HHI lower bound over log-regulations
is 2.07, which corresponds to a p-value of 0.041. The reduced form
regression of inventor share over log restriction change is equally
highly significant. Accordingly, the SW underidentification test rejects
the null hypothesis at a 5\% confidence level. Given the weakness
of the instruments, I also report the Anderson-Rubin p-value, which
confirms that the coefficient is 5\% significant, and the corresponding
confidence intervals in brackets.

Taken together, the results presented in this section establish a
causal link between the increase in inventor concentration and the
shifts in product market concentration across NAICS 4-digit sectors.

\begin{table}
\caption{IV Regressions of Change in 4-digit Knowledge Market Share over Change
in HHI Lower Bound, 2SLS Long-Difference, 1997-2012\label{tab: RegIV}}

\begin{centering}
\subfloat[2SLS Results]{
\centering{}\scalebox{.9}{{
\def\sym#1{\ifmmode^{#1}\else\(^{#1}\)\fi}
\begin{tabular}{l*{2}{c}}
\hline\hline
                    &$\Delta$ Inventor Share (pp)   &               \\
                    &\multicolumn{1}{c}{(1)}   &\multicolumn{1}{c}{(2)}   \\
\hline
$\Delta \underline{\text{HHI}}$&      32.426+  &      30.096+  \\
                    &    (16.987)   &    (15.819)   \\
                    &$\left[4.850,\  99.013\right]$   &$\left[4.415,\  92.104\right]$   \\
$\Delta \log$ Sales &               &       0.525*  \\
                    &               &     (0.247)   \\
\hline
Knowledge Market FE &   \ding{51}   &   \ding{51}   \\
Sample              & Full Sample   &Mahalanobis 5\%   \\
Weight              &       Sales   &       Sales   \\
Observations        &         157   &         150   \\
First-Stage F       &    4.656786   &    4.753009   \\
Anderson-Rubin p-value&    .0298009   &    .0321185   \\
\hline\hline
\end{tabular}
}
}}\\
\subfloat[First Stage and Reduced Form]{
\centering{}\scalebox{1}{{
\def\sym#1{\ifmmode^{#1}\else\(^{#1}\)\fi}
\makebox[\linewidth][c]{
	\begin{tabular}{lcc}
\hline\hline
                    &$\Delta$ Inventor Share (pp)   &$\Delta \underline{HHI}$   \\
                    &\multicolumn{1}{c}{(1)}   &(2)  \\
\hline
$\Delta \log$ Restrictions&       0.478*  &       0.016*  \\
                    &     (0.220)   &     (0.007)   \\
$\Delta \log$ Sales   &       0.539+  &      -0.000   \\
                    &     (0.274)   &     (0.005)   \\
\hline
Knowledge Market FE&   \ding{51}   &   \ding{51}   \\
Sample              & Full Sample   & Full Sample   \\
Weight              &       Sales   &       Sales   \\
Observations        &         153   &         153   \\
\hline\hline
\end{tabular}
}
}
}}\\
\par\end{centering}
\raggedright{}{\small{}Note: Regressions weighted by sales in 2012;
Robust standard errors in parentheses; Symbols denote significance
levels $\left(+\ p<0.1,^{*}\ p<0.05,^{**}\ p<.01,^{***}\ p<.001\right)$;
Checkmarks indicate the inclusion of fixed effects. These Tables presents
the results of specifications (\ref{eq: spec}), when the outcome
is the share of effective inventors of sector $p$ over total inventors
in knowledge market $k$, and the independent variable is the change
in the lower bound of the Herfindal-Hirschman Index for product market
$p$, as implied by Economic Census concentration ratios, instrumented
by the change in log-restrictions relevant to the NAICS sector. The
lower panel present first-stage and reduced-form relations. ``Full
Sample'' and ``Mahalanobis 5\%'' refer to the samples described
in the main text.}{\small\par}
\end{table}


\subsubsection{Sectors that Attracted More Researchers Saw Increasing Top Firms'
Inventor Shares and Falling Patent Forward Citations}

While the findings presented so far establish a connection between
inventor and product market concentration, they do not establish that
changes in the distribution of researchers across sectors are inefficient.
In particular, it would not be unreasonable to think that more concentrated
sectors saw increased entry as a result of the higher rents captured
by incumbents. Table \ref{tab: RegStats} shows that the opposite
is occurring. Specifically, the share of effective inventors accruing
to top inventor-hiring firms has increased in the sectors that attracted
more inventors over the period considered, relative to firms with
less inventors in the sector. This finding is consistent across a
variety of measures displayed in Columns (1) to (6). These findings
suggest that inventors have increasingly concentrated among large
incumbents, that is, sectors that increased their inventor share also
saw a \emph{within-sector} increase in inventor concentration. 

Throughout this section, I present results using changes in inventor
shares to focus directly on the correlation between inventor transitions
and their within-sector distribution. Unless otherwise noted, these
findings are robust to using the change in the HHI rather than the
inventor share, as should be expected from the strong correlation
between these two variables reported in previous tables. For this
section, and other patent-based measure, I present robustness results
employing this alternative specification in Appendix \ref{subsec:Patents-with-HHI}.

My next finding suggests that this trend might be driven by a rise
in defensive innovation, that is R\&D aimed at sheltering the incumbents'
dominant position and raising barriers to entry. Table \ref{tab: RegFwdCite}
shows that inventors' concentration in specific sectors has gone hand
in hand with a fall in patents' forward citations in these sectors,
a standard measure of the impact of patents and their role in paving
the way for further innovations \citep{hallNBERPatentCitation2001}.
The result in Columns (1) and (2) report two different measures of
forward citations, that differ in how the series are corrected for
truncation. As discussed in Section \ref{subsec:Other data and aggregation},
one measure (Column (2)) uses the procedure delineated by \citet{hallNBERPatentCitation2001},
computing the forward citation lag distribution conditioning on the
technology class of the cited patent, while the other (Column (1))
also conditions on the technology class of citing patents. Column
(3) presents the estimates relative to patent generality, another
measure of patent impact, which increases with the scope of application
of the patent. The regressions in this table are unweighted since
they rely only on patent data, but results are robust to using the
HHI as a regressor and weighting by sales. I present results for the
full sample, as well as restricting to the middle range of changes
in inventor shares, which contains more than 90\% of the observations.
In both samples, I find a highly significant negative relation between
changes in inventor shares and the fall in forward citations. The
coefficients imply a high semi-elasticity of self citations to changes
in the inventor shares, whereby a 1pp increase in the share of inventors
leads to an average 0.2-0.5\% reduction in forward citations. When
dropping extreme observations, I also find a significant decrease
in the generality of the patents, indicating that concentrating sectors
produce less widely applicable patents. This effect is however not
robust to estimating the regression using the HHI as the independent
variable. 

The fall in forward citations is a first indication of the presence
of defensive innovation, aimed primarily at barring entrants from
developing new technologies \citep[see, e.g.,][]{guellecPreemptivePatentingSecuring2012}.
In the next section, I show that these patents also appear to produce
limited benefits in terms of ensuing productivity growth, as measured
by output per worker growth. This provides further support to my interpretation.

Before moving to the results on productivity, I investigate a competing
explanation for my findings on output growth. As highlighted by \citet{acemogluRadicalIncrementalInnovationForthcoming}
and \citet{akcigitGrowthHeterogeneousInnovations2018} among others,
large incumbents have a strong incentive to focus on improving their
own products at the expense of broadly applicable innovation. In the
words of the authors, internal and incremental innovation prevails
on more radical, external innovations. This mechanism would also imply
that an increase in incumbents' share of R\&D resources leads to falling
innovation productivity. In order to asses the importance of this
channel, and in keeping with the analysis in \citet{akcigitGrowthHeterogeneousInnovations2018},
I use the share of self-citations to measure the extent of internal
innovation conducted by firms. Table \ref{tab: RegSelfCite} displays
the results pertaining to this measure. All columns use as dependent
variable the change in excess log self-citations as defined in Section
\ref{subsec:Other data and aggregation}. Columns (1) and (2) build
excess self-citations correcting for the importance of firms' patents
for the CPC group, which reflects the technological classification
of the patent. Columns (3) and (4) use the more narrowly-defined CPC
subgroups for robustness. Coefficients are mostly non-significant,
and turn negative when knowledge market fixed effects are included.
Column (3) displays a marginally significant coefficient. However,
this result is not robust to using the HHI as regressor and weighting
regressions by sales as in the baseline specification. The findings
in this table suggest that incremental innovation does changes significantly
following changes in sector concentration, reducing the scope for
this alternative explanation. 

\begin{sidewaystable}[ph]
\caption{Regressions of Change in Inventor Distribution Measures over Change
in 4-digit Knowledge Market Share, Long-Difference, 1997-2012\label{tab: RegStats}}

\begin{centering}
\scalebox{1}{{
\def\sym#1{\ifmmode^{#1}\else\(^{#1}\)\fi}
\begin{tabular}{l*{5}{c}}
\hline\hline
                    &Ch. Inv. 90/50 Quantile Ratio   &Ch. Inv. Top-10/Bottom-50 Share Ratio   &Ch. Inv. Top-50/Bottom-50 Share Ratio   &Ch. Inv. Top 10\% Share   &Ch. Inv. Bottom 50\% Share   \\
                    &\multicolumn{1}{c}{(1)}   &\multicolumn{1}{c}{(2)}   &\multicolumn{1}{c}{(3)}   &\multicolumn{1}{c}{(4)}   &\multicolumn{1}{c}{(5)}   \\
\hline
Ch. 4d K.M. Eff. Inv. Share (\%)&       0.211+  &       0.243*  &       0.314+  &       0.018** &      -0.008*  \\
                    &     (0.107)   &     (0.097)   &     (0.184)   &     (0.006)   &     (0.004)   \\
Ch. Log Real Sales  &      -0.100   &       0.328   &       0.147   &       0.026   &       0.005   \\
                    &     (0.122)   &     (0.294)   &     (0.316)   &     (0.020)   &     (0.007)   \\
\hline
4D Knowledge Market FE&   \ding{51}   &   \ding{51}   &   \ding{51}   &   \ding{51}   &   \ding{51}   \\
Sample              & Full Sample   & Full Sample   & Full Sample   & Full Sample   & Full Sample   \\
Weight              &       Sales   &       Sales   &       Sales   &       Sales   &       Sales   \\
Observations        &         118   &         118   &         118   &         118   &         118   \\
\hline\hline
\end{tabular}
}
}\\
\par\end{centering}
\raggedright{}{\small{}Note: Regressions weighted by sales in 2012;
Robust standard errors in parentheses; Symbols denote significance
levels $\left(+\ p<0.1,^{*}\ p<0.05,^{**}\ p<.01,^{***}\ p<.001\right)$;
Checkmarks indicate the inclusion of fixed effects. Please refer to
notes in Table \ref{tab: RegShInvHHI} for further details. Column
(1) uses the ratio in the 90 percentile of effective inventors to
the median as the outcome variable. Columns (2) and (3) instead present
the share ratio, that is the share of effective inventors accruing
to the top 10 or 50\% relative to the share accruing to the bottom
50\% of the distribution within each NAICS sector.}{\small\par}
\end{sidewaystable}

\begin{table}
\caption{Regressions of Changes in Forward Citation over 4-digit Knowledge
Market Share, Long-Differences, 1997-2012\label{tab: RegFwdCite}}

\begin{centering}
\subfloat[Full sample]{\begin{centering}
\par\end{centering}
\centering{}\scalebox{.9}{{
\def\sym#1{\ifmmode^{#1}\else\(^{#1}\)\fi}
\begin{tabular}{l*{3}{c}}
\hline\hline
                    &Ch. in log citations per patent (CPC2 based)   &Ch. in log citations per patent (Total)   &Ch. in patent generality   \\
                    &\multicolumn{1}{c}{(1)}   &\multicolumn{1}{c}{(2)}   &\multicolumn{1}{c}{(3)}   \\
\hline
Ch. 4d K.M. Eff. Inv. Share (\%)&      -0.197***&      -0.227***&      -0.004   \\
                    &     (0.044)   &     (0.051)   &     (0.004)   \\
Ch. Log Real Sales  &      -0.234*  &      -0.258+  &       0.008   \\
                    &     (0.112)   &     (0.148)   &     (0.013)   \\
\hline
4D Knowledge Market FE&   \ding{51}   &   \ding{51}   &   \ding{51}   \\
Sample              & Full Sample   & Full Sample   & Full Sample   \\
Weight              &               &               &               \\
Observations        &         153   &         153   &         153   \\
\hline\hline
\end{tabular}
}
}}
\par\end{centering}

\begin{centering}
\subfloat[Full sample, restricting to the middle range of the change in inventor
shares ($-2\%$ to $+2\%$)]{\centering{}\scalebox{.9}{ {
\def\sym#1{\ifmmode^{#1}\else\(^{#1}\)\fi}
\begin{tabular}{l*{3}{c}}
\hline\hline
                    &$\Delta \log$ Citations/Patent (CPC)   &$\Delta \log$ Citations/Patent (Total)   &$\Delta$ Patent Generality   \\
                    &\multicolumn{1}{c}{(1)}   &\multicolumn{1}{c}{(2)}   &\multicolumn{1}{c}{(3)}   \\
\hline
$\Delta$ Inventor Share (pp)&      -0.545***&      -0.618***&      -0.025*  \\
                    &     (0.113)   &     (0.137)   &     (0.012)   \\
$\Delta \log$ Sales &      -0.232*  &      -0.255+  &       0.008   \\
                    &     (0.109)   &     (0.146)   &     (0.012)   \\
\hline
Knowledge Market FE &   \ding{51}   &   \ding{51}   &   \ding{51}   \\
Sample              & Full Sample   & Full Sample   & Full Sample   \\
Weight              &               &               &               \\
Observations        &         144   &         144   &         144   \\
\hline\hline
\end{tabular}
}
}}\\
\par\end{centering}
\raggedright{}{\small{}Note: Unweighted regressions; Robust standard
errors in parentheses; Symbols denote significance levels $\left(+\ p<0.1,^{*}\ p<0.05,^{**}\ p<.01,^{***}\ p<.001\right)$;
Checkmarks indicate the inclusion of fixed effects. These Tables presents
the results of specification (\ref{eq: spec}), when the outcome is
the log-change in forward citations and the change in patent generality
in sector $p$ over the change in the share of inventors employed
in sector $p$. Column (1) and (2) presents the results when forward
citations are extrapolated the procedure Hall et al. (2000) to avoid
truncation bias. A specific cite-lag distribution over 35 years is
estimated for each pair of cited and citing CPC2-codes. Column (1)
employs the extrapolation scheme by each pair of CPC2 cited and citing
sector. Column (2) applies the extrapolation scheme to total citations
received by each cited patent. Column (3) presents results on the
patent generality measures. All columns exclude self-citations. Upper
panel: full sample; Bottom panel: excluding sectors with absolute
increase in the inventor share above 2\%.}{\small\par}
\end{table}
\begin{table}
\caption{Regressions of Change in Excess Self-Citations over 4-digit Knowledge
Market Share, Long-Differences, 1997-2012\label{tab: RegSelfCite}}

\begin{centering}
\scalebox{.9}{{
\def\sym#1{\ifmmode^{#1}\else\(^{#1}\)\fi}
\begin{tabular}{l*{4}{c}}
\hline\hline
                    &Ch. excess self-c. over CPC group   &               &Ch. excess self-c.  over CPC subgroup   &               \\
                    &\multicolumn{1}{c}{(1)}   &\multicolumn{1}{c}{(2)}   &\multicolumn{1}{c}{(3)}   &\multicolumn{1}{c}{(4)}   \\
\hline
Ch. 4d K.M. Eff. Inv. Share (\%)&       0.920   &      -0.444   &       0.958+  &      -0.228   \\
                    &     (0.711)   &     (1.083)   &     (0.512)   &     (0.801)   \\
Ch. Log Real Sales  &      -1.841   &      -1.954   &      -1.456   &      -1.674   \\
                    &     (1.925)   &     (1.988)   &     (1.326)   &     (1.279)   \\
\hline
4D Knowledge Market FE&               &   \ding{51}   &               &   \ding{51}   \\
Sample              & Full Sample   & Full Sample   & Full Sample   & Full Sample   \\
Weight              &               &               &               &               \\
Observations        &         157   &         153   &         157   &         153   \\
\hline\hline
\end{tabular}
}
}
\par\end{centering}
\begin{centering}
\par\end{centering}
\raggedright{}{\small{}Note: Unweighted regressions; Robust standard
errors in parentheses; Symbols denote significance levels $\left(+\ p<0.1,^{*}\ p<0.05,^{**}\ p<.01,^{***}\ p<.001\right)$;
Checkmarks indicate the inclusion of fixed effects. This Tables presents
the results of specifications (\ref{eq: spec}), when the outcome
is the change in excess self-citations in sector $p$ over the change
in the share of inventors employed in sector $p$. }{\small\par}
\end{table}
\FloatBarrier

\subsubsection{Markets with Growing Inventor Shares Experienced a Fall in Inventor
Productivity}

Table \ref{tab: RegProd} presents the results of running regression
(\ref{eq: spec}) when the outcome is the average growth in output
per worker per effective inventor. I use growth in annual output per
worker provided by the Economic Census and average this measure over
the five-year window starting in the Economic Census year, and I proceed
analogously to build a measure of average effective inventors over
the same period. Inventor productivity is then defined as average
output per worker growth divided by average effective inventors. Both
the outcome and the dependent variable are measured in percentage
points. Table \ref{tab: RegProd} reveals a negative and significant
correlation between the increase in the effective inventors' change
and inventor productivity, regardless of the independent variable
employed and the sample restriction adopted.

The magnitude of estimated coefficients can be grasped considering
the scale of the variables and their changes over the sample period.
Starting from the upper panel of Table \ref{tab: RegProd}, the median
change in the share of effective inventors over the period was .014pp,
while the measure of effective inventors has a median of 2018.\footnote{Recall that effective inventors in each year are measured as the sum
of inventor fixed-effects in each year, and therefore do not represent
the simple count of inventors.} Using the coefficient in Column (5) to predict the median annual
change in labor productivity growth implied by rising inventor concentration
amounts to a fall of .15pp ($-.005\times.014pp\times2018$). This
number increases to . 28pp when using the statistics relative to sectors
with positive growth in labor productivity only, which accounted for
the bulk of the increase in inventor shares. An alternative back-of-the-envelope
computation, using the change in product market concentration to predict
the change in inventor shares gives even starker results. Using the
coefficient in Column (2) of Table \ref{tab: RegShInvHHI}(a), and
considering a median change in the HHI of 0.002 yields an increase
in the share of effective inventors in concentrating sectors of 0.045pp,
which implies a fall in average labor productivity implied by misallocation
of $0.45$pp. While these numbers might appear sizable considering
the entirety of the economy, it is worth noting that the sample I
have data for includes mainly manufacturing and retail sectors, which
experienced a sizable reduction of about $2.8$pp in average annual
productivity growth from 1997-2012, driven by a steep decline in output
per worker growth in manufacturing. Therefore, the mechanism I propose
would explain around 15\% of the observed decrease in output per worker
growth in these sectors.

The estimates in the lower panel of Table \ref{tab: RegProd}, which
uses the HHI instead the change in inventor shares as independent
variable, imply even larger growth effects. Using the estimates in
Column (2), a median HHI change of 0.02 and a median number of effective
inventors of 1421 in sectors with growth in inventor shares implies
a $-0.78$pp change in output per worker growth from misallocation,
with a confidence interval ranging from $-0.13$ to $-1.45$pp. The
midpoint of these estimates would explain $27\%$ of the observed
fall in output per worker growth over the sample period, with bounds
ranging from around $5\%$ to about $50\%$.

This last set of results provides further support to the hypothesis
that defensive innovation increased in concentrating sectors. Indeed,
a central feature of this type of R\&D activity is that it does not
produce substantial growth, since it is aimed at hampering innovation
projects by potential entrants. In this sense, patents are registered
in order to prevent others from doing so and implementing inventions
that would threaten the incumbents' dominant position. This interpretation
also resonates with the finding in \citet{argentePatentsProductsProduct2020a},
who note that incumbent firms tend to register a large number of patents,
but account for a small share of overall innovations; a finding they
also interpret as evidence of defensive innovation.

\begin{table}[h]
\caption{Regressions of Changes in Inventor Productivity over Changes in Inventors'
Share and HHI, Long-Difference, 1997-2012\label{tab: RegProd}}

\begin{centering}
\subfloat[Change in Inventors' Share as Independent Variable]{
\centering{}\scalebox{.95}{{
\def\sym#1{\ifmmode^{#1}\else\(^{#1}\)\fi}
\begin{tabular}{l*{4}{c}}
\hline\hline
                    &$\Delta$ Growth/Inventor (pp)   &               &               &               \\
                    &\multicolumn{1}{c}{(1)}   &\multicolumn{1}{c}{(2)}   &\multicolumn{1}{c}{(3)}   &\multicolumn{1}{c}{(4)}   \\
\hline
$\Delta$ Inventor Share (pp)&      -0.007** &      -0.005*  &      -0.007** &      -0.005*  \\
                    &     (0.002)   &     (0.002)   &     (0.002)   &     (0.002)   \\
$\Delta \log$ Sales &               &      -0.051*  &               &      -0.054*  \\
                    &               &     (0.021)   &               &     (0.021)   \\
\hline
Knowledge Market FE &   \ding{51}   &   \ding{51}   &   \ding{51}   &   \ding{51}   \\
Sample              & Full Sample   & Full Sample   &Mahalanobis 5\%   &Mahalanobis 5\%   \\
Weight              &       Sales   &       Sales   &       Sales   &       Sales   \\
Observations        &         101   &         101   &          96   &          93   \\
\hline\hline
\end{tabular}
}
}}\\
\subfloat[Change in HHI as Independent Variable]{
\centering{}\scalebox{.95}{\input{\string"/Users/andreamanera/Dropbox (MIT)/Research/ProductInnovation/docs/JMP/draft/tab/Lprod_hhi_inv_fe1.tex\string"}}}\\
\par\end{centering}
\raggedright{}{\small{}Note: Regressions weighted by sales in 2012;
Robust standard errors in parentheses; Symbols denote significance
levels $\left(+\ p<0.1,^{*}\ p<0.05,^{**}\ p<.01,^{***}\ p<.001\right)$;
Checkmarks indicate the inclusion of fixed effects. Please refer to
notes in Table \ref{tab: RegShInvHHI} for further details. Inventor
productivity is measured as the average growth in output per worker
over the five years starting in the Economic Census year over the
total number of effective inventors in each sector. The upper panel
presents estimates when the independent variable is the change in
the share of inventors accruing to a sector, while the bottom panel
uses the change in the lower bound of the HHI index.}{\small\par}
\end{table}
\FloatBarrier
