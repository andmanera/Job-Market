%% LyX 2.1.2 created this file.  For more info, see http://www.lyx.org/.
%% Do not edit unless you really know what you are doing.
\documentclass[12pt,english]{article}
\usepackage[T1]{fontenc}
\usepackage[latin9]{inputenc}
\usepackage{geometry}
\geometry{verbose,tmargin=2.5cm,bmargin=2.5cm,lmargin=2cm,rmargin=2cm}
\setcounter{secnumdepth}{3}
\setcounter{tocdepth}{3}
\usepackage{amsthm}
\usepackage{amsmath}
\usepackage{amssymb}
\usepackage{graphicx}
\usepackage{setspace}
\usepackage{esint}
\onehalfspacing

\makeatletter

%%%%%%%%%%%%%%%%%%%%%%%%%%%%%% LyX specific LaTeX commands.
%% Special footnote code from the package 'stblftnt.sty'
%% Author: Robin Fairbairns -- Last revised Dec 13 1996
\let\SF@@footnote\footnote
\def\footnote{\ifx\protect\@typeset@protect
    \expandafter\SF@@footnote
  \else
    \expandafter\SF@gobble@opt
  \fi
}
\expandafter\def\csname SF@gobble@opt \endcsname{\@ifnextchar[%]
  \SF@gobble@twobracket
  \@gobble
}
\edef\SF@gobble@opt{\noexpand\protect
  \expandafter\noexpand\csname SF@gobble@opt \endcsname}
\def\SF@gobble@twobracket[#1]#2{}

%%%%%%%%%%%%%%%%%%%%%%%%%%%%%% Textclass specific LaTeX commands.
\usepackage{enumitem}		% customizable list environments
\newlength{\lyxlabelwidth}      % auxiliary length 
\theoremstyle{plain}
\newtheorem{thm}{\protect\theoremname}
  \theoremstyle{definition}
  \newtheorem{defn}[thm]{\protect\definitionname}
  \theoremstyle{plain}
  \newtheorem*{prop*}{\protect\propositionname}

\makeatother

\usepackage{babel}
  \providecommand{\definitionname}{Definition}
  \providecommand{\propositionname}{Proposition}
\providecommand{\theoremname}{Theorem}

\begin{document}
\title{Recitation 8: Expected Utility}
\author{Andrea Manera\footnote{I thank Alex He for sharing his notes. Remaining errors are my own.}}
\date{Spring 2020}
\maketitle




\section{Willingness to pay for insurance}

\noindent Suppose that a risk-averse agent:
\begin{itemize}
\item has von Neumann-Morgenstern utility function given by $u(w)=\ln(w)$,
\item faces a 50 percent chance of getting $w=\$200$ and a complementary
50 percent chance of having $w=\$100$.\\
\end{itemize}
\textbf{Question: How much is the agent willing to pay, in expected
value, to avoid this uncertainty?}
\begin{itemize}
\item In absence of insurance, the agent's expected utility is:
\[
E[u(w)]=0.5\ln200+0.5\ln100=4.952.
\]
\item Expected wealth is:
\[
E[w]=0.5\times200+0.5\times100=\$150.
\]
\item The certainty equivalent (i.e., the fixed amount of money the gives
the same expected utility as the uncertain realization of the agent's
wealth) is:
\[
CE=\exp[4.952]=\$141.5,
\]
because \$141.5 given to the agent with probability = 1 yield the
same expected utility as the lottery described above.
\item Thus, to avoid uncertainty, the agent is willing to pay:
\[
WTP=E[w]-CE=\$150-\$141.5=\$8.50.
\]
\end{itemize}
The hard point to conceptualize in this example (and in the relevant
question on the problem set) is that, given the uncertainty, ex-ante
we don't know the exact realization of the agent's wealth (i.e., whether
she'll have \$200 or \$100). Thus, the agent's willingness to pay
for insurance has to be ``benchmarked'' to her expected wealth.
Specifically, her willingness to pay is given by the difference between
her expected wealth absent insurance and her certainty equivalent.



%%%%%%%%%%%%%%%%%%%%%%%%%%%%%%%%%%%%%%%%
%%%%%%%%%%%%%%%%%%%%%%%%%%%%%%%%%%%%%%%%
%%%%%%%%%%%%%%%%%%%%%%%%%%%%%%%%%%%%%%%%
%%%%%%%%%%%%%%%%%%%%%%%%%%%%%%%%%%%%%%%%
%%%%%%%%%%%%%%%%%%%%%%%%%%%%%%%%%%%%%%%%
%%%%%%%%%%%%%%%%%%%%%%%%%%%%%%%%%%%%%%%%


\section{Deriving Expected Utility Theorem}



\begin{defn}
A simple lottery $L$ is a list $L=\left(p_{1},...p_{N}\right)$ with
$p_{n}\geq0$ for all $n$ and $\Sigma_{n}p_{n}=1,$ where $p_{n}$
is interpreted as the probability of outcome $n$ occurring. 
\end{defn}

 Take the set of alternatives the
decision maker faces, denoted by $\pounds$ to be the set of all simple
lotteries over possible outcomes $N$.
We assume the consumer has a rational preference relation $\succsim$
on $\pounds,$ a \emph{complete} and \emph{transitive} relation allowing
comparison among any pair of simple lotteries.

\vskip2ex
 \textbf{Axiom 1. Continuity}\emph{. Small changes in probabilities
do not change the nature of the ordering of two lotteries. This can
be made concrete here (I won't use formal notation b/c it's a mess).
If a ``bowl of miso soup\textquotedblright{} is preferable to a ``cup
of Kenyan coffee,\textquotedblright{} then a mixture of the outcome
``bowl of miso soup\textquotedblright{} and a sufficiently small
but positive probability of ``death by sushi knife\textquotedblright{}
is still preferred to ``cup of Kenyan coffee.\textquotedblright{}}
 
\vskip2ex
\textbf{Axiom 2. Independence.} \emph{The preference relation }$\succsim$\emph{\ on
the space of simple lotteries }$\pounds$\emph{ satisfies the independence
axiom if for all }$L,L^{\prime},L^{\prime\prime}\in\pounds$\emph{
and }$\alpha\in\left(0,1\right)$\emph{, we have} 
\[
L\succsim L^{\prime}\text{ if and only if ~}\alpha L+\left(1-\alpha\right)L^{\prime\prime}\succsim\alpha L^{\prime}+\left(1-\alpha\right)L^{\prime\prime}\text{. }
\]

 In words, when we mix each of two lotteries with a third one, then
the preference ordering of the two resulting mixtures does not depend
on (is \emph{independent of}) the particular third lottery used.
 Example: If a bowl of miso soup is preferred to cup of Peets coffee,
then the lottery (bowl of miso soup with 50\% probability, steak dinner
with 50\% probability) is preferred to the lottery (cup of Peets coffee
with 50\% probability, steak dinner with 50\% probability). 

\subsection{Expected utility theory}
\begin{itemize}
\item We now want to define a class of utility functions over risky choices
that have the ``expected utility form.\textquotedblright{} We will
then prove that if a utility function satisfies the definitions above
for \emph{continuity }and \emph{independence} in preferences over
lotteries, then the utility function has the expected utility form.

\begin{defn}
\emph{The utility function }$U:\pounds\rightarrow\mathbb{R}$\emph{
has an expected utility form if there is an assignment of numbers
}$\left(u_{1},...u_{N}\right)$\emph{ to the }$N$\emph{ outcomes
such that for every simple lottery }$L=\left(p_{1},...,p_{N}\right)\in\pounds$\emph{
we have that }
\[
U\left(L\right)=u_{1}p_{1}+...+u_{N}p_{N}.
\]

\end{defn}
\item A utility function with the expected utility form is called a Von
Neumann-Morgenstern (VNM) expected utility function.

\item In other words, a utility function has the expected utility form if
and only if:
\[
U\left(\sum{}_{k=1}^{K}\alpha_{k}L_{k}\right)=\sum{}_{k=1}^{K}\alpha_{k}U\left(L_{k}\right)
\]
for any $K$ lotteries $L_{k}\in\pounds$, $k=1,...,K,$ and probabilities
$\left(\alpha_{1},...,\alpha_{K}\right)\geq0$, $\Sigma_{k}\alpha_{k}=1.$
\item Intuitively, a utility function has the expected utility property
if the utility of a lottery is simply the (probability) weighted average
of the utility of each of the outcomes.

\end{itemize}

\subsection{Proof of expected utility property}
\begin{prop*}
  (Expected utility theory) Suppose that the rational preference
  relation $\succsim$ on the space of lotteries $\pounds$ satisfies
  the continuity and independence axioms. Then $\succsim$ admits a
  utility representation of the expected utility form. That is, we can
  assign a number $u_{n}$ to each outcome $n=1,...,N$ in such a manner
  that for any two lotteries $L=\left(p_{1},...,p_{N}\right)$ and
  $L^{\prime}=\left(p_{1}^{\prime},...p_{N}^{\prime}\right),$ we have
  $L\succsim L^{\prime}$ if and only if \[
\sum\limits _{n=1}^{N}u_np_{n}\geq\sum\limits _{n=1}^{N}u_{n}p_{n}^{\prime}
\].
\end{prop*}

\begin{proof}
  We will show that for any two lotteries $L$ and $L'$, and $\beta\in
  (0,1)$, there is a utility function $U$ representing preferences
  over lotteries, such that $U(\beta L + (1-\beta) L') = \beta U(L) +
  (1-\beta) U(L')$. This is equivalent to showing the Expected Utility
  Property stated above because if we take $L_n$ to be a lottery that
  results in outcome $n$ with certainty, then $U(L) = U(\sum_np_nL_n)
  = \sum_n p_nU(L_n) = \sum_n p_nu_n$


Expected Utility Property (in five steps)
\end{proof}
\qquad{}Assume that there are best and worst lotteries in $\pounds$,
$\bar{L}$ and \underline{$L$}.
\begin{enumerate}
\item If $L$ $\succ L^{\prime}$ and $\alpha\in\left(0,1\right),$ then
$L\succ\alpha L+\left(1-\alpha\right)L^{\prime}\succ L^{\prime}.$
This follows immediately from the independence axiom.
\item Let $\alpha,\beta\in\left[0,1\right].$ Then $\beta\bar{L}+\left(1-\beta\right)$\underline{$L$}$\succ\alpha\bar{L}+\left(1-\alpha\right)$\underline{$L$}
if and only if $\beta>\alpha$. This follows from the prior step.
\item For any $L\in\pounds$, there is a unique $\alpha_{L}$ such that
$\left[\alpha_{L}\bar{L}+\left(1-\alpha_{L}\right)\underline{L}\right]\sim L$.
Existence follows from continuity. Uniqueness follows from the prior
step.
\item We now need to define a utility function that satisfies the
  expected utility property. Though there may be many choices, our
  proposition only requires us to pick one that represents the
  preferences over lotteries and satisfies the expected utility
  property. Condsider the function $U:\pounds\rightarrow\mathbb{R}$
  that assigns $U\left(L\right)=\alpha_{L}$ for all $L\in\pounds$. It
  represents the preference relation $\succsim$ because from Step 3,
  we know that for any two lotteries $L,L^{\prime}\in\pounds,$ we have
\[
L\succsim L^{\prime}\text{ if and only if }\left[\alpha_{L}\bar{L}+\left(1-\alpha_{L}\right)\underline{L}\right]\succsim\left[\alpha_{L^{\prime}}\bar{L}+\left(1-\alpha_{L^{\prime}}\right)\underline{L}\right].
\]
Thus $L\succsim L^{\prime}$ if and only if $\alpha_{L}\geq\alpha_{L^{\prime}}$.
\item The utility function $U\left(\cdot\right)$ that assigns $U\left(L\right)=\alpha_{L}$
for all $L\in\pounds$ is linear and therefore has the expected utility
form. \\
\textbf{We want to show that for any }$L,L^{\prime}\in\pounds,$\textbf{\
and }$\beta\left[0,1\right],$\textbf{\ we have }$U\left(\beta L+\left(1-\beta\right)L^{\prime}\right)=\beta U\left(L\right)+\left(1-\beta\right)U\left(L^{\prime}\right).$
\\
By step (3) above, we have 
\begin{eqnarray*}
L & \sim & U\left(L\right)\bar{L}+\left(1-U\left(L\right)\right)\underline{L}=\alpha_{L}\bar{L}+\left(1-\alpha_{L}\right)\underline{L}\\
L^{\prime} & \sim & U\left(L^{\prime}\right)\bar{L}+\left(1-U\left(L^{\prime}\right)\right)\underline{L}=\alpha_{L^\prime}\bar{L}+\left(1-\alpha_{L^\prime}\right)\underline{L}.
\end{eqnarray*}
By the Independence Axiom, 
\begin{eqnarray*}
\beta L+\left(1-\beta\right)L^{\prime} &\sim & \beta\left[U\left(L\right)\bar{L}+\left(1-U\left(L\right)\right)\underline{L}\right]+\left(1-\beta\right)\left[U\left(L^{\prime}\right)\bar{L}+\left(1-U\left(L^{\prime}\right)\right)\underline{L}\right]\\
&  = & \left[\beta U\left(L\right)+\left(1-\beta\right)U\left(L^{\prime}\right)\right]\bar{L}+\left[\beta\left(1-U\left(L\right)\right)+\left(1-\beta\right)\left(1-U\left(L\right)^{\prime}\right)\right]\underline{L}\\
 & = & \left[\beta U\left(L\right)+\left(1-\beta\right)U\left(L^{\prime}\right)\right]\bar{L}+\left[1-\beta U\left(L\right)+\left(\beta-1\right)U\left(L^{\prime}\right)\right]\underline{L}.
\end{eqnarray*}
By step (4), this expression can be written as 
\begin{eqnarray*}
 &  & \left[\beta\alpha_{L}+\left(1-\beta\right)\alpha_{L^{\prime}}\right]\bar{L}+\left[1-\beta\alpha_{L}+\left(\beta-1\right)\alpha_{L^\prime}\right]\underline{L}\\
 & = & \beta U\left(L\right)+\left(1-\beta\right)U\left(L^{\prime}\right).
\end{eqnarray*}
This establishes that a utility function that satisfies continuity
and the Independence Axiom, has the expected utility property: $U\left(\beta L+\left(1-\beta\right)L^{\prime}\right)=\beta U\left(L\right)+\left(1-\beta\right)U\left(L^{\prime}\right)$ 
\end{enumerate}



\end{document}
