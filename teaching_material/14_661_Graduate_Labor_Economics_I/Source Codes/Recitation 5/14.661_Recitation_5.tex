%% LyX 2.3.5.2 created this file.  For more info, see http://www.lyx.org/.
%% Do not edit unless you really know what you are doing.
\documentclass[english]{article}
\usepackage{fourier}
\usepackage[T1]{fontenc}
\usepackage[latin9]{inputenc}
\usepackage{geometry}
\geometry{verbose}
\setlength{\parskip}{\medskipamount}
\setlength{\parindent}{0pt}
\usepackage{calc}
\usepackage{units}
\usepackage{bm}
\usepackage{amsmath}

\makeatletter
%%%%%%%%%%%%%%%%%%%%%%%%%%%%%% User specified LaTeX commands.
\usepackage{babel}

\usepackage{geometry}
%\geometry{verbose,tmargin=1in,bmargin=1in,lmargin=1in,rmargin=1in,headheight=1in,headsep=1in,footskip=1in}
\setlength{\parskip}{\medskipamount}
\setlength{\parindent}{0pt}
\usepackage{units}
\usepackage{amsmath}
\usepackage{babel}
\newcommand{\Der}[2]{\frac{\partial #1}{\partial #2}}
\usepackage{pifont}

\makeatother

\usepackage{babel}
\begin{document}
\title{14.661 Recitation 5}
\author{Andrea Manera\thanks{These notes draw from the ones prepared by Cl�mence Idoux, who drew
on Christopher Walters' (former TA, now Berkeley Professor), which
in turn are based on Hamermesh's (not a former TA) HOLE chapter on
labor demand.}}
\date{October 22, 2021}

\maketitle
Today we will derive the \textit{fundamental law of labor demand},
which allows us to express the long run elasticity of labor demand
in terms of (1) the elasticity of substitution between inputs ($\sigma$),
(2) the elasticity of demand for the final product ($\eta$), (3)
the labor's share of income ($s_{L}$).

\textbf{Fundamental Law of Labor Demand:} \textit{Suppose that (i)
there is a perfectly competitive market for a final good with demand
$D\left(p\right)$ and (ii) producers have access to a CRTS production
function $F(K,L)$. Then the elasticity of labor demand ($\eta_{L}$)
is:}

\begin{equation}
\eta_{L}\equiv\Der{\log L(w,r)}{\log w}=\underbrace{-\sigma(1-s_{L})}_{\text{substitution effect}}\underbrace{-\eta s_{L}}_{\text{scale effect}}\label{key}
\end{equation}

where 
\begin{itemize}
\item $\sigma$ is the elasticity of substitution between labor and capital,
defined to be positive $\sigma\equiv\bm{-}\Der{\log(K/L)}{\log(r/w)}$ 
\item $s_{L}$ is the labor share: $s_{L}\equiv\frac{wL}{PY}$ 
\item $\eta$ is the final good demand elasticity $\eta\equiv D'\left(p\right)\frac{p}{D\left(p\right)}$ 
\end{itemize}
We will derive this result by building on several individual results,
we show in the first sections.

\section*{Review of Properties of Homogeneous Functions}

A function $f(x_{1},...,x_{N})$ defined for all nonnegative values
$(x_{1},...,x_{N})\geq0$ is said to be homogeneous of degree $k$,
if for every $\alpha>0$,
\begin{equation}
f(\alpha x_{1},...,\alpha x_{N})=\alpha^{k}f(x_{1},...,x_{N}).\label{key}
\end{equation}

Homogeneous functions' derivatives have two interesting properties,
that we can get using the implicit function theorem. Recall the IFT.
For an implicit function for $y(x)$ defined so that:
\[
G(x,y)=0,
\]
as long as the function is differentiable in $x$ and $y$ and the
derivative wrt $y$ is non-zero, we have:
\[
\rightarrow\frac{\mathrm{d}y}{\mathrm{d}x}=-\frac{\nicefrac{\partial G}{\partial x}}{\nicefrac{\partial G}{\partial y}}.
\]
Now consider:
\[
G(\bm{x},f(\bm{x}))=f(\alpha x_{1},...,\alpha x_{N})-\alpha^{k}f(x_{1},...,x_{N})
\]
Then:
\begin{align*}
\alpha\frac{\partial f(\alpha x_{1},...,\alpha x_{N})}{\partial x_{1}}\mathrm{dx}-\alpha^{k}\mathrm{d}f(x_{1},...,x_{N}) & =0,\\
\frac{\partial f(\alpha x_{1},...,\alpha x_{N})}{\partial x_{1}} & =\alpha^{k-1}\frac{\partial f(x_{1},...,x_{N})}{\mathrm{\partial x}},
\end{align*}
so the derivative is homogenous of degree $k-1$!

Another cool trick (aka Euler's homogeneous function theorem) by using
the IFT wrt $\alpha$:
\begin{align*}
f(\alpha x_{1},...,\alpha x_{N})-\alpha^{k}f(x_{1},...,x_{N})\\
\sum_{i=1}^{N}x_{1}\frac{\partial f(\alpha x_{1},...,\alpha x_{N})}{\partial x}\mathrm{d}\alpha-k\alpha^{k-1}f(x_{1},...,x_{N})\mathrm{d}\alpha
\end{align*}
Setting $\alpha=1$, gives Euler's formula
\begin{align*}
\sum_{i=1}^{N}x_{i}\Der{f(\alpha x_{1},...,\alpha x_{N})}{x_{i}} & =kf(x_{1},...,x_{N})
\end{align*}


\paragraph{Formal Recap}
\begin{center}
\fbox{\begin{minipage}[c]{6in}%
\textbf{Property 1 (HF1):} If $f(x_{1},...,x_{N})$ is homogeneous
of degree $k$, for $k=...-1,0,1....$ then $\Der{f(x_{1},...,x_{N})}{x_{i}}$
is homogeneous of degree $k-1$ for all $i=1,...N$%
\end{minipage}}
\par\end{center}

\begin{center}
\fbox{\begin{minipage}[c]{6in}%
\textbf{Property 2 (Euler's homogeneous function theorem):} If $f(x_{1},...,x_{N})$
is homogeneous of degree $k$, for $k=...-1,0,1....$ then $\sum_{i=1}^{N}\Der{f(x_{1},...,x_{N})}{x_{i}}x_{i}=kf(x_{1},...,x_{N})$%
\end{minipage}}
\par\end{center}

\section*{Applications to Cost Minimization}

Dual of profit maximization is cost minimization:

\begin{align*}
c(w,r,q)=\min_{K,L}wL+rK\quad\text{s.t.}\quad F(K,L)\geq q
\end{align*}

Just like the expenditure minimization from consumer theory, with
same properties. 

First and foremost, Shephard's Lemma (aka Envelope Theorem).

Second, The cost function is homogeneous of degree 1 in \emph{input
prices }(double all prices, I do not change my optimal relative choices
but cost doubles):

\begin{align*}
c(\alpha w,\alpha r,q)= & \min_{K,L}\alpha wL+\alpha rK\quad\text{s.t.}\quad F(K,L)\geq q\\
= & \alpha\min_{K,L}wL+rK\quad\text{s.t.}\quad F(K,L)\geq q\\
= & \alpha c(w,r,q)
\end{align*}
Third, the cost function is homogeneous in the \emph{quantity to produce,
$q,$}\textbf{ assuming constant returns to scale} (homogeneity of
degree 1 of the production function). In this case, the optimal capital/labor
ratio is fixed regardless of $q$, and only depends on $\left(w,r\right)$.
In math:

\begin{align*}
c(w,r,\alpha q)= & \min_{K,L}wL+rK\quad\text{s.t.}\quad F(K,L)\geq\alpha q,\\
= & \alpha\min_{K,L}w\frac{L}{\alpha}+r\frac{K}{\alpha}\quad\text{s.t.}\quad F\left(\frac{K}{\alpha},\frac{L}{\alpha}\right)\geq q,\\
= & \alpha c(w,r,q).
\end{align*}
In particular:
\[
c(w,r,q)=qc(w,r,1)\equiv qc^{U}(w,r),
\]
where $c^{U}$ denotes the \emph{unit cost function.}

\paragraph{Formal Recap}
\begin{center}
\fbox{\begin{minipage}[c]{6in}%
\textbf{Result 1 (Shephard's Lemma):} 
\begin{align*}
\dfrac{\partial c(w,r,q)}{\partial w}=L^{c}(w,r,q),\quad\dfrac{\partial c(w,r,q)}{\partial r}=K^{c}(w,r,q),
\end{align*}
Where $\left(L^{c}(w,r,q),K^{c}(w,r,q)\right)=\arg\min_{\left(L,K\right)}wL+rK\quad\text{s.t.}\quad F(K,L)\geq q$%
\end{minipage}}\\
 %
\fbox{\begin{minipage}[c]{6in}%
\textbf{Result 2:} \textit{For $\alpha\neq0$,}

\[
c(\alpha w,\alpha r,q)=\alpha c(w,r,q)
\]
%
\end{minipage}}\\
\fbox{\begin{minipage}[c]{6in}%
\textbf{Result 3:} \textit{For $\alpha\neq0$,}

\[
c(w,r,\alpha q)=\alpha c(w,r,q).
\]
In particular,
\[
c(w,r,q)=qc^{U}(w,r)
\]
%
\end{minipage}}
\par\end{center}

\subsection*{Elasticity of Substitution}

The elasticity of substitution $\sigma$ for a production function
is defined as\footnote{Trivia: the elasticity of substitution was first defined in Hicks'
\emph{Theory of Wages}, which came out in 1932. This was the same
year of too many horrible and tragic events: the Nazi party won the
elections in Germany, Japan occupies Manchuria and restore the last
Chinese Emperor Pu Yi as a puppet, the start of the Great Famine in
Ukraine, the start of the fascist dictatorship in Portugal... But
some good things happened too, in addition to Hicks' books: Aldous
Axley published Brave New World, FDR won the election, and Goofy made
his first appearance! Also, Australia declared war on Emus (check
this one out!)}
\begin{center}
$\sigma\equiv\bm{-}\dfrac{\mathrm{d}\log\left(\nicefrac{K^{c}}{L^{c}}\right)}{\mathrm{d}\log\left(\nicefrac{r}{w}\right)}=\dfrac{\mathrm{d}\log\left(\nicefrac{K^{c}}{L^{c}}\right)}{\mathrm{d}\log\left(\nicefrac{w}{r}\right)}.$ 
\par\end{center}

This is just \emph{minus }the elasticity of the capital-labor ratio
to relative factor prices $r/w$. We put a minus because the elasticity
is clearly negative (also properties of the cost minimization problem).
When the production function has constant returns to scale (CRS): 
\begin{center}
\fbox{\begin{minipage}[c]{6in}%
 \textbf{Result 4:} 
\[
\sigma=\dfrac{c^{U}c_{wr}^{U}}{c_{r}^{U}c_{w}^{U}}.
\]
%
\end{minipage}} 
\par\end{center}

This result relies heavily on the homogeneity of the cost function.
By what we have see above:
\begin{enumerate}
\item $c(w,r,q)=wc_{w}+rc_{r}$, from Shephard' Lemma.
\item $c_{w}(w,r,q)=c_{w}+wc_{ww}+rc_{rw}\Rightarrow wc_{ww}+rc_{rw}=0$.
\item $c$ is homogeneous of degree 1 $\Rightarrow$ $c_{w},c_{r}$ are
homogeneous of degree $0$ $\Rightarrow c_{ww},c_{rw}$ are homogeneous
of degree $-1$.
\end{enumerate}
Combining these fact:

\begin{align*}
\log\left(\nicefrac{K^{c}}{L^{c}}\right)= & \log c_{r}(w,r,q)-\log c_{w}(w,r,q)\\
= & \log c_{r}\left(\frac{w}{r},1,q\right)-\log c_{w}\left(\frac{w}{r},1,q\right)
\end{align*}

where the second line uses the fact that $c_{w},c_{r}$, are homogeneous
of degree 0 in factor prices.

Note:

\begin{align*}
\sigma= & \dfrac{\mathrm{d}\log\left(\nicefrac{K^{c}}{L^{c}}\right)}{\mathrm{d}\log\left(\nicefrac{w}{r}\right)}=\dfrac{\partial\log\left(\nicefrac{K^{c}}{L^{c}}\right)}{\partial\left(\nicefrac{w}{r}\right)}\cdot\dfrac{w}{r}\\
= & \dfrac{w}{r}\left(\dfrac{c_{rw}\left(\frac{w}{r},1,q\right)}{c_{r}\left(\frac{w}{r},1,q\right)}-\dfrac{c_{ww}\left(\frac{w}{r},1,q\right)}{c_{w}\left(\frac{w}{r},1,q\right)}\right)\\
= & \dfrac{w}{r}\left(\dfrac{rc_{rw}(w,r,q)}{c_{r}(w,r,q)}-\dfrac{rc_{ww}(w,r,q)}{c_{w}(w,r,q)}\right)
\end{align*}

where the last line uses the fact that $c_{ww}$, $c_{rw}$, are homogenous
of degree -1 in $(w,r)$.

By $wc_{ww}+rc_{wr}=0$:

\begin{align*}
\dfrac{w}{r}=-\dfrac{c_{wr}}{c_{ww}}\quad\text{and}\quad wc_{ww}=-rc_{wr}.
\end{align*}

Thus, finally:

\begin{align*}
\sigma= & -\dfrac{c_{wr}}{c_{ww}}\left(-\dfrac{wc_{ww}}{c_{r}}-\dfrac{rc_{ww}}{c_{w}}\right)\\
= & c_{wr}\left(\dfrac{wc_{w}+rc_{r}}{c_{r}c_{w}}\right)\\
= & \dfrac{c\cdot c_{wr}}{c_{r}c_{w}}\\
= & \dfrac{c^{U}\cdot c_{wr}^{U}}{c_{r}^{U}c_{w}^{U}},
\end{align*}

where the third line uses Euler's HFT and the last the fact that $c$is
homogeneous of degree 1, $c_{wr}$ of degree -1, and $c_{r},c_{w}$
of degree $0$.

\paragraph{Formal Recap}
\begin{center}
\fbox{\begin{minipage}[c]{6in}%
\textbf{Result 4:} 
\[
\bm{-}\dfrac{\mathrm{d}\log\left(\nicefrac{K^{c}}{L^{c}}\right)}{\mathrm{d}\log\left(\nicefrac{r}{w}\right)}\equiv\sigma=\dfrac{c^{U}c_{wr}^{U}}{c_{r}^{U}c_{w}^{U}}.
\]
%
\end{minipage}} 
\par\end{center}

\section*{Market Demand}
\begin{center}
\fbox{\begin{minipage}[c]{6in}%
 \textbf{Result 5:} 
\[
L=Qc_{w}^{U}\text{and}\quad K=Qc_{r}^{U}
\]
%
\end{minipage}} 
\par\end{center}

Assuming all firms have the same production function, we can thus
write the market's demand for labor as:

\begin{align*}
L(w,r)= & \sum_{j}L_{j}^{c}(w,r,q_{j})\\
= & \sum_{j}c_{w}(w,r,q_{j})\quad & \text{By R1}\\
= & \sum_{j}q_{j}c_{w}^{U}(w,r)\quad & \text{By R4}\\
= & Qc_{w}^{U}(w,r)
\end{align*}

where $Q$ is the total quantity produced in the market. A similar
type of reasoning leads to the equivalent result for capital demand.

The main result in this section gives us a convenient expression for
the derivative of the market demand for labor $L(w,r)$, in terms
of substitution and a scale effect.
\begin{center}
\fbox{\begin{minipage}[c]{6in}%
 \textbf{Result 6:} 
\[
\frac{\partial L(w,r)}{\partial w}=D\left(p\right)c_{ww}^{U}+D'\left(p\right)\left(c_{w}^{U}\right){}^{2}
\]
%
\end{minipage}} 
\par\end{center}

Begin by noticing that in equilibrium $Q$ will have to be equal to
$D\left(p\right)$, so that

\begin{align*}
L(w,r)= & D\left(p\right)c_{w}^{U}(w,r)
\end{align*}

Furthermore, in competitive equilibrium it must be the case that the
product price is equal to marginal cost for each firm, $p=c^{u}(w,r)$,
so that we can write the previous expression as follows:

\begin{align*}
L(w,r)= & D(c^{U}(w,r))c_{w}^{U}(w,r)
\end{align*}

We can now differentiate this expression to obtain the result above.

\section*{Finally: The Fundamental Law of Factor Demand}

Let's put everything together:

\begin{align*}
\frac{\partial L(w,r)}{\partial w}= & D\left(p\right)c_{ww}^{U}+D'\left(p\right)\left(c_{w}^{U}\right)^{2}\\
= & -D(c^{U}(w,r))\dfrac{r}{w}c_{wr}^{U}+D'\left(p\right)\left(c_{w}^{U}\right)^{2}\quad & \text{By Euler's HFT}\\
= & -D\left(p\right)\cdot\dfrac{r}{w}\cdot\sigma\cdot\dfrac{c_{r}^{U}c_{w}^{U}}{c^{U}}+D'\left(p\right)\left(c_{w}^{U}\right)^{2}\quad & \text{By R4}\\
= & -D\left(p\right)\cdot\dfrac{r}{w}\cdot\dfrac{\sigma}{c^{U}}\cdot\dfrac{LK}{Q^{2}}+D'\left(p\right)\cdot\dfrac{L^{2}}{Q^{2}}\quad & \text{By R5}\\
= & -\dfrac{rK}{pQ}\cdot\dfrac{\sigma L}{w}+D'\left(p\right)\cdot\dfrac{L^{2}}{D\left(p\right)}
\end{align*}

were we used that $c^{U}=p$ and $D\left(p\right)=Q$ in the last
step.

We now write this as an elasticity:

\begin{align*}
\eta_{L}= & \dfrac{\partial L}{\partial w}\cdot\dfrac{w}{L}\\
= & -\dfrac{rK}{pQ}\cdot\dfrac{\sigma L}{w}\cdot\dfrac{w}{L}+D'\left(p\right)\cdot\dfrac{L^{2}}{D\left(p\right)}\cdot\dfrac{w}{L}\\
= & -s_{K}\cdot\sigma+\dfrac{D'\left(p\right)}{D\left(p\right)}\cdot\dfrac{wL}{D\left(p\right)}\quad & \text{Since }s_{K}=\dfrac{rK}{pQ}\\
= & =-s_{K}\cdot\sigma+D'\left(p\right)\cdot\dfrac{p}{D\left(p\right)}\cdot\dfrac{wL}{pQ}\\
= & -s_{K}\cdot\sigma-\eta\cdot s_{L}\quad & \text{Since }s_{L}=\dfrac{wL}{pQ}\\
= & -(1-s_{L})\cdot\sigma-\eta\cdot s_{L}
\end{align*}

In the last step we used $s_{K}+s_{L}=1$ since there are no profits
(or, alternatively, it is implied by Euler's HFT along with $p=c^{U}$).

\subsection*{Comparative Statics}
\begin{enumerate}
\item $\sigma\uparrow\implies\vert\eta_{L}\vert\uparrow$: When substitution
to other factors is easier, the market demand curve for labor is more
elastic. Intuition: use more capital in response to higher wages
\item $\eta\uparrow\implies\vert\eta_{L}\vert\uparrow$: When product demand
is more elastic, labor demand is more elastic. Intuition: I loose
more demand when I have to pass on higher wages to prices, so I also
produce less.
\item $\left(s_{L}\uparrow\implies\vert\eta_{L}\vert\uparrow\right)\Longleftrightarrow\left(\sigma<\eta\right)$:
When labor's share of income increases, labor demand becomes more
elastic iff the elasticity of product demand is larger than the elasticity
of substitution. Intuition: if the labor share is high, it means $WL$
is relatively high, that is I employ lots of labor. If the elasticity
of product demand is very large, I loose lots of demand (labor) when
the wage increases. However, if I have a high elasticity of substitution,
I can trade off (relatively little) labor to save demand and charge
not so high prices, which result in a lower decrease in labor demand.
\end{enumerate}
These are three of the ``Hicks-Marshall Laws of Derived Demand.''
The remaining one requires looking at the elasticity of \emph{supply}
of other factors, which here we did not cover, which would affect
how much easy it is to scale up with capital relative to labor. That
is:
\begin{enumerate}
\item[4.]  $\vert\eta_{L}\vert\uparrow$ if the supply of other factors is
more elastic. Intuition: it is easier to scale up the use of alternative
factors.
\end{enumerate}

\end{document}
